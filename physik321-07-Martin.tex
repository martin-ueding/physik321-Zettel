% Copyright © 2012 Martin Ueding <dev@martin-ueding.de>
%
\documentclass[11pt, ngerman, fleqn]{article}

\usepackage[a4paper, left=3cm, right=2cm, top=2cm, bottom=2cm]{geometry}
\usepackage[activate]{pdfcprot}
\usepackage[cdot, squaren]{SIunits}
\usepackage[iso]{isodate}
\usepackage[parfill]{parskip}
\usepackage[T1]{fontenc}
\usepackage[utf8]{inputenc}
\usepackage{amsmath}
\usepackage{amsthm}
\usepackage{babel}
\usepackage{color}
\usepackage{commath}
\usepackage{fancyhdr}
\usepackage{graphicx}
\usepackage{hyperref}
\usepackage{lastpage}
\usepackage{setspace}
\usepackage{tikz}

\usepackage[charter, greekuppercase=italicized]{mathdesign}

\definecolor{darkblue}{rgb}{0,0,.5}
\definecolor{darkgreen}{rgb}{0,.5,0}

\hypersetup{
	breaklinks=false,
	citecolor=darkgreen,
	colorlinks=true,
	linkcolor=black,
	menucolor=black,
	urlcolor=darkblue,
}

\setlength{\columnsep}{2cm}

\DeclareMathOperator{\arcsinh}{arsinh}
\DeclareMathOperator{\arsinh}{arsinh}
\DeclareMathOperator{\asinh}{arsinh}
\DeclareMathOperator{\card}{card}
\DeclareMathOperator{\diam}{diam}

\newcommand{\dalambert}{\mathop{{}\Box}\nolimits}
\newcommand{\divergence}[1]{\inner{\vnabla}{#1}}
\newcommand{\ee}{\mathrm e}
\newcommand{\emesswert}{\del{\messwert \pm \messwert}}
\newcommand{\ev}{\hat{\vec e}}
\newcommand{\e}[1]{\cdot 10^{#1}}
\newcommand{\fehlt}{\textcolor{red}{Hier fehlen noch Inhalte.}}
\newcommand{\half}{\frac 12}
\newcommand{\ii}{\mathrm i}
\newcommand{\inner}[2]{\left\langle #1, #2 \right\rangle}
\newcommand{\laplace}{\mathop{{}\Deltaup}\nolimits}
\newcommand{\messwert}{\textcolor{blue}{\square}}
\newcommand{\punkte}{\textcolor{white}{xxxxx}}
\newcommand{\tens}[1]{\boldsymbol{\mathsf{#1}}}
\newcommand{\vnabla}{\vec \nabla}
\renewcommand{\vec}[1]{\boldsymbol{#1}}

\newcommand{\themodul}{physik321}
\newcommand{\thegruppe}{Gruppe 8 -- Julia Volmer}
\newcommand{\theuebung}{7}

\pagestyle{fancy}

\fancyfoot[C]{\footnotesize{\thegruppe}}
\fancyfoot[L]{\footnotesize{Martin Ueding, Simon Schlepphorst}}
\fancyfoot[R]{\footnotesize{Seite \thepage\ / \pageref{LastPage}}}
\fancyhead[L]{\themodul{} -- Übung \theuebung}

\def\thesection{H \theuebung.\arabic{section}}
\def\thesubsubsection{\thesubsection\alph{section}}

\title{\themodul{} -- Übung \theuebung \\ \vspace{0.5cm} \large{\thegruppe}}

\author{
	Martin Ueding \\ \small{\href{mailto:mu@uni-bonn.de}{mu@uni-bonn.de}}
	\and
	Simon Schlepphorst \\ \small{\href{mailto:s2@uni-bonn.de}{s2@uni-bonn.de}}
}

\begin{document}

\maketitle

\begin{table}[h]
	\centering
	\begin{tabular}{l|c|c|c|c}
		Aufgabe & \ref 1 & \ref 2 & \ref 3 & $\sum$   \\
		\hline
		Punkte & \punkte / 20 & \punkte / 10 & \punkte / 10 & \punkte / 40
	\end{tabular}
\end{table}

%%%%%%%%%%%%%%%%%%%%%%%%%%%%%%%%%%%%%%%%%%%%%%%%%%%%%%%%%%%%%%%%%%%%%%%%%%%%%%%
%                         Reflexion und Transmission                          %
%%%%%%%%%%%%%%%%%%%%%%%%%%%%%%%%%%%%%%%%%%%%%%%%%%%%%%%%%%%%%%%%%%%%%%%%%%%%%%%

\section{Reflexion und Transmission}
\label 1

\subsection{Brechungsgesetz, Amplituden}

\paragraph{Einfallswinkel}

Es soll gerade gelten, dass für alle $\vec r$ entlang der Grenzschicht die gleiche Phase herrscht. Somit muss gelten:
\begin{align*}
	\exp\del{i\del{\inner{\vec k}{\vec r} - \omega t}} &= \exp\del{i\del{\inner{\vec k''}{\vec r} - \omega t}} \\
	\intertext{Daraus folgt:}
	\inner{\vec k}{\vec r} &= \inner{\vec k''}{\vec r} \\
	  k r \cos\del{\alpha} &= k'' r \cos\del{\alpha ''} \\
	\intertext{Da allerdings gerade $k = k''$ gilt, folgt:}
	\alpha &= \alpha''
\end{align*}

Somit ist Einfalls- gleich Ausfallswinkel.

\paragraph{Brechungsgesetz}

Wir setzen analog an:
\begin{align*}
	\exp\del{i\del{\inner{\vec k}{\vec r} - \omega t}} &= \exp\del{i\del{\inner{\vec k'}{\vec r} - \omega t}} \\
	  k \cos\del{\alpha} &= k' \cos\del{\alpha '} \\
	  \sqrt{\varepsilon \mu} \cos\del{\alpha} &= \sqrt{\varepsilon' \mu'} \cos\del{\alpha '} \\
	  \frac{\cos\del{\alpha}}{\cos\del{\alpha '}} &= \frac{\sqrt{\varepsilon' \mu'}}{\sqrt{\varepsilon \mu}} \\
	  \frac{\cos\del{\alpha}}{\cos\del{\alpha '}} &= \frac{n'}{n}
	\intertext{%
		Allerdings ist $\alpha$ der Winkel zwischen Strahl und Grenzfläche. Wir
		setzen $\theta = \pi/2 - \alpha$ ein und erhalten das Brechungsgesetz:
	}
	  \frac{\sin\del{\theta}}{\sin\del{\theta '}} &= \frac{n'}{n}
\end{align*}

\paragraph{Amplituden}

Es sollen die vier Relationen hergeleitet werden.

An der Grenzfläche gilt:
\[
	\vnabla \times \vec E + \dot{\vec B} = 0
	, \quad
	\divergence{\vec D} = 0
\]

\begin{enumerate}
	\item
		Daraus können wir ableiten, dass die zur Grenzfläche parallele
		Komponente stetig sein muss. Die parallelen Komponenten erhalten wir
		durch Projektion der zur Ebene parallelen Komponenten mit
		$\cos\del{\theta}$ und $\cos\del{\theta'}$:
		\[
			\del{E_\parallel - E_\parallel''} \cos\del{\theta} - E_\parallel' \cos\del{\theta'} = 0
		\]

	\item \fehlt

	\item
		Die senkrechte Komponente muss genauso wie in der ersten Relation
		stetig sein. Dabei müssen wir hier allerdings nicht projizieren. Somit
		gilt direkt:
		\[
			E_\parallel - E_\parallel'' - E_\parallel' = 0
		\]

	\item \fehlt
\end{enumerate}

\subsection{Reflexions- und Transmissionsfaktor}

\fehlt

%%%%%%%%%%%%%%%%%%%%%%%%%%%%%%%%%%%%%%%%%%%%%%%%%%%%%%%%%%%%%%%%%%%%%%%%%%%%%%%
%                     Reflexion von unpolarisiertem Licht                     %
%%%%%%%%%%%%%%%%%%%%%%%%%%%%%%%%%%%%%%%%%%%%%%%%%%%%%%%%%%%%%%%%%%%%%%%%%%%%%%%

\section{Reflexion von unpolarisiertem Licht}
\label 2

\fehlt

%%%%%%%%%%%%%%%%%%%%%%%%%%%%%%%%%%%%%%%%%%%%%%%%%%%%%%%%%%%%%%%%%%%%%%%%%%%%%%%
%                    Isolator im elektromagnetischen Feld                     %
%%%%%%%%%%%%%%%%%%%%%%%%%%%%%%%%%%%%%%%%%%%%%%%%%%%%%%%%%%%%%%%%%%%%%%%%%%%%%%%

\section{Isolator im elektromagnetischen Feld}
\label 3

\subsection{Maxwellgleichungen}

Im Isolator mit $\mu_r$ und $\varepsilon_r$ können wir $\vec D$ und $\vec H$
ersetzen und erhalten:
\[
	\divergence{\vec E} = 0
	,\quad
	\divergence{\vec B} = 0
	,\quad
	\frac 1{\mu_r \mu_0} \vnabla \times \vec B - \varepsilon_r \varepsilon_0 \dot{\vec E} = \vec 0
	,\quad
	\vnabla \times \vec E + \dot{\vec B} = \vec 0
\]

\subsection{homogene Wellengleichung}

Wir stellen die letzte Gleichung um, bilden die Rotation der Dritten und
erhalten:
\[
	\ddot{\vec B}
	= - \vnabla \times \dot{\vec E}
	= - c^2 \vnabla \times \del{\vnabla \times \vec B}
	= - c^2 \del{\vnabla \divergence{\vec B} - \laplace \vec B}
		= c^2 \laplace \vec B
\]

Somit gilt $\dalambert \vec B = \vec 0$.

\subsection{Berechnung des magnetischen Flusses}
\label c

Das $\vec E$-Feld breitet sich in $z$-Richtung aus. Somit muss das $\vec
B$-Feld dies auch tun. Außerdem handelt es sich hier um ebene Wellen, so dass
$\partial_x \vec B = \vec 0$ und $\partial_y \vec B = \vec 0$ gelten muss. Wir
setzen das gegebene elektrische Feld dritte Gleichung ein und erhalten als
Differentialgleichung:
\[
	c^2
	\begin{pmatrix}
		- \partial_z B_y \\ \partial_z B_x \\ 0
	\end{pmatrix}
	=
	\frac{E_0}{5}
	\begin{pmatrix}
		1 \\ -2 \\ 0
	\end{pmatrix}
	\exp\del{\ii \del{kz-\omega t}} (-\ii \omega)
\]

Die ersten beiden Zeilen können wir integrieren und erhalten (von
einer additiven harmonischen Funktion abgesehen):
\[
	\vec B = \frac{1}{c^2} \frac{\omega}{k} \frac{E_0}{5}
	\begin{pmatrix}
		2 \\ 1 \\ 0
	\end{pmatrix}
	\exp\del{\ii \del{kz-\omega t}}
\]

Die Polarisierung ist linear. Dabei sind $\vec E$- und $\vec B$-Feld und
$\pi/2$ phasenverschoben, das letztere vor dem ersteren (siehe Abbildung
\ref{fig:3-3}). Beide Felder schwingen in der Ebene, die von ihrem Feld und der
$z$-Achse aufgespannt wird.

\begin{figure}
	\centering
	\begin{tikzpicture}
		\draw[thick, ->] (0, 0) -- (-60:2) node[right] {$\vec E$};
		\draw[thick, ->] (0, 0) -- (30:2) node[right] {$\vec B$};
		\draw[dotted, ->] (0, 0) -- (120:2) node[left] {$\vec E'$};
		\draw[dotted, ->] (0, 0) -- (210:2) node[left] {$\vec B'$};
		\draw[->] (-2, 0) -- (2, 0) node[right] {$x$};
		\draw[->] (0, -2) -- (0, 2) node[above] {$y$};
	\end{tikzpicture}
	\caption{Polarisation der Felder in Aufgabe \ref c}
	\label{fig:3-3}
\end{figure}

\subsection{Berechnung des elektrischen Feldes}
\label d

Wir setzen analog zur Aufgabe \ref c an und erhalten als Differentialgleichung:
\[
	\begin{pmatrix}
		- \partial_z E_y \\ \partial_z E_x \\ 0
	\end{pmatrix}
	=
	B_0 \omega
	\begin{pmatrix}
		\sin\del{kz-\omega t} \\ - \cos\del{kz-\omega t} \\ 0
	\end{pmatrix}
\]

Durch Integration erhalten wir (bis auf eine harmonische Funktion):
\[
	\vec E = B_0 \frac \omega k
	\begin{pmatrix}
		- \sin\del{kz-\omega t} \\ \cos\del{kz-\omega t} \\ 0
	\end{pmatrix}
\]

Diese Welle ist zirkulär polarisiert. Dabei ist das $\vec E$- um $\pi/2$ vor
dem $\vec B$-Feld. Ein bestimmter Zeitpunkt ist in Abbildung \ref{fig:3-4}
gezeigt.

\begin{figure}
	\centering
	\begin{tikzpicture}
		\draw[thick, ->] (0, 0) -- (30:2) node[right] {$\vec B$};
		\draw[dotted, ->] (30:1.5) arc (30:50:1.5);
		\draw[thick, ->] (0, 0) -- (120:2) node[right] {$\vec E$};
		\draw[dotted, ->] (120:1.5) arc (120:140:1.5);
		\draw[->] (-2, 0) -- (2, 0) node[right] {$x$};
		\draw[->] (0, -2) -- (0, 2) node[above] {$y$};
	\end{tikzpicture}
	\caption{Polarisation der Felder in Aufgabe \ref d}
	\label{fig:3-4}
\end{figure}

%\bibliography{../../zentrale_BibTeX/Central}
%\bibliographystyle{plain}

\end{document}

% vim: spell spelllang=de
