% Copyright © 2012 Martin Ueding <dev@martin-ueding.de>
%
\documentclass[11pt, ngerman, fleqn]{article}

\usepackage[a4paper, left=3cm, right=2cm, top=2cm, bottom=2cm]{geometry}
\usepackage[activate]{pdfcprot}
\usepackage[cdot, squaren]{SIunits}
\usepackage[iso]{isodate}
\usepackage[parfill]{parskip}
\usepackage[T1]{fontenc}
\usepackage[utf8]{inputenc}
\usepackage{amsmath}
\usepackage{amsthm}
\usepackage{babel}
\usepackage{color}
\usepackage{commath}
\usepackage{fancyhdr}
\usepackage{graphicx}
\usepackage{hyperref}
\usepackage{lastpage}
\usepackage{setspace}
\usepackage{tikz}

\usepackage[charter, greekuppercase=italicized]{mathdesign}

\definecolor{darkblue}{rgb}{0,0,.5}
\definecolor{darkgreen}{rgb}{0,.5,0}

\hypersetup{
	breaklinks=false,
	citecolor=darkgreen,
	colorlinks=true,
	linkcolor=black,
	menucolor=black,
	urlcolor=darkblue,
}

\setlength{\columnsep}{2cm}

\DeclareMathOperator{\arcsinh}{arsinh}
\DeclareMathOperator{\arsinh}{arsinh}
\DeclareMathOperator{\asinh}{arsinh}
\DeclareMathOperator{\card}{card}
\DeclareMathOperator{\diam}{diam}

\newcommand{\dalambert}{\mathop{{}\Box}\nolimits}
\newcommand{\divergence}[1]{\inner{\vnabla}{#1}}
\newcommand{\ee}{\mathrm e}
\newcommand{\emesswert}{\del{\messwert \pm \messwert}}
\newcommand{\ev}{\hat{\vec e}}
\newcommand{\e}[1]{\cdot 10^{#1}}
\newcommand{\fehlt}{\textcolor{red}{Hier fehlen noch Inhalte.}}
\newcommand{\half}{\frac 12}
\newcommand{\ii}{\mathrm i}
\newcommand{\inner}[2]{\left\langle #1, #2 \right\rangle}
\newcommand{\laplace}{\mathop{{}\Deltaup}\nolimits}
\newcommand{\messwert}{\textcolor{blue}{\square}}
\newcommand{\punkte}{\textcolor{white}{xxxxx}}
\newcommand{\tens}[1]{\boldsymbol{#1}}
\newcommand{\vnabla}{\vec \nabla}
\renewcommand{\vec}[1]{\boldsymbol{#1}}

\newcommand{\themodul}{physik321}
\newcommand{\thegruppe}{Gruppe 8 -- Julia Volmer}
\newcommand{\theuebung}{3}

\pagestyle{fancy}

\fancyfoot[C]{\footnotesize{\thegruppe}}
\fancyfoot[L]{\footnotesize{Martin Ueding, Simon Schlepphorst}}
\fancyfoot[R]{\footnotesize{Seite \thepage\ / \pageref{LastPage}}}
\fancyhead[L]{\themodul{} -- Übung \theuebung}

\setcounter{section}{0}

\def\thesection{H \theuebung.\arabic{section}}
\def\thesubsubsection{\thesubsection\alph{section}}

\title{\themodul{} -- Übung \theuebung \\ \vspace{0.5cm} \large{\thegruppe}}

\author{Martin Ueding \\ \small{\href{mailto:mu@uni-bonn.de}{mu@uni-bonn.de}} \and Simon Schlepphorst \\ \small{\href{mailto:s2@uni-bonn.de}{s2@uni-bonn.de}}}

\begin{document}

\maketitle

\begin{table}[h]
	\centering
	\begin{tabular}{l|c|c|c|c|c}
		Aufgabe & H \theuebung.1 & H \theuebung.2 & H \theuebung.3 & H \theuebung.4 & $\sum$   \\
		\hline
		Punkte & \punkte / 10 & \punkte / 10 & 0 / 20 & \punkte / 20 & \punkte / 40
	\end{tabular}
\end{table}

%%%%%%%%%%%%%%%%%%%%%%%%%%%%%%%%%%%%%%%%%%%%%%%%%%%%%%%%%%%%%%%%%%%%%%%%%%%%%%%
%                    eine Konfiguration von Punktladungen                     %
%%%%%%%%%%%%%%%%%%%%%%%%%%%%%%%%%%%%%%%%%%%%%%%%%%%%%%%%%%%%%%%%%%%%%%%%%%%%%%%

\section{eine Konfiguration von Punktladungen}

\subsection{Ladungsverteilung und Potential}

Das Potential einer Punktladung ist:
\[
	\Phi\del{\vec r} = \frac{1}{4\pi\varepsilon_0} \frac{Q}{\abs{\vec r - \vec r'}}
\]

Die Ladungsverteilung kann ich mit $\delta$-Distributionen darstellen:
\[
	\rho\del{\vec r} = Q \delta\del{x} \delta\del{y} \del{
		\delta\del{\frac a2 - z} - \delta\del{-\frac a2 - z}
	}
\]

Die Punktladungen haben jeweils ein Potential, das ich mit Superposition zu
einem Gesamtpotential zusammensetzen kann:
\[
	\Phi\del{\vec r} = \frac{1}{4 \pi \varepsilon_0} \del{
		\frac{Q}{\abs{\vec r - \begin{pmatrix}
	0 \\ 0 \\ a/2
	\end{pmatrix}}} - \frac{Q}{\abs{\vec r - \begin{pmatrix}
	0 \\ 0 \\ - a/2
	\end{pmatrix}}}
	}
\]

\subsection{Ladungen zusammenrücken}

Die Dipolmoment $p = Qa$ soll konstant bleiben, dabei soll $a \to 0$ gehen.
Ich entwickele das Potential um $a = 0$:
\[
	\Phi\del{\vec r} = \frac{1}{4 \pi \varepsilon_0} \frac{a Q z}{r^3} + \mathcal O\del{a^2}
	= \frac{1}{4 \pi \varepsilon_0} \frac{pz}{r^3} + \mathcal O\del{a^2}
\]

Wenn man jetzt ein Einheitensystem wählt, in dem $1/\del{4 \pi \varepsilon_0} = 1$
gilt, dann ergibt dies das geforderte $\Phi = pz/r^3$.

\subsection{andere Ladungsdichte}

Gegeben ist die Ladungsdichte:
\[
	\rho\del{\vec r} = - p \delta(x) \delta(y) \delta'(z)
\]

Durch Integration erhalte ich das Potential:
\begin{align*}
	\Phi\del{\vec x}
	&= - \frac{1}{4 \pi \varepsilon_0} \int_{\mathbb R} \dif x' \int_{\mathbb R} \dif y' \int_{\mathbb R} \dif z'  \frac{\rho\del{x', y', z'}}{\abs{\vec r - \vec r'}} \\
	&= - \frac{1}{4 \pi \varepsilon_0} p \int_{\mathbb R} \dif x' \delta\del{x'} \int_{\mathbb R} \dif y' \delta\del{y'} \int_{\mathbb R} \dif z' \frac{\delta'\del{z'}}{\abs{\vec r - \vec r'}} \\
	\intertext{%
		Durch partielle Integration kann ich die Ableitung nach $z$ von der
		$\delta$-Distribution auf den inversen Abstand umwälzen.
	}
	&= - \frac{1}{4 \pi \varepsilon_0} p \int_{\mathbb R} \dif x' \delta\del{x'} \int_{\mathbb R} \dif y' \delta\del{y'} \int_{\mathbb R} \dif z' \delta\del{z'} \dpd{}{z'} \frac{1}{\abs{\vec r - \vec r'}} \\
	&= - \frac{1}{4 \pi \varepsilon_0} p \int_{\mathbb R} \dif x' \delta\del{x'} \int_{\mathbb R} \dif y' \delta\del{y'} \int_{\mathbb R} \dif z' \frac{\delta\del{z'} \del{z - z'}}{\del{\del{x-x'}^2 + \del{y-y'}^2 + \del{z-z'}^2}^{3/2}} \\
	\intertext{%
		Die Integrale sind jetzt recht einfach durch die
		$\delta$-Distributionen. Letztlich ist $x' = 0$, $y' = 0$, $z' = 0$.
	}
	&= - \frac{1}{4 \pi \varepsilon_0} p \frac{z}{\del{x^2 + y^2 + z^2}^{3/2}} \\
	&= - \frac{1}{4 \pi \varepsilon_0} p \frac{z}{r^3}
\end{align*}

Das ist genau das gleiche Potential wie bei der vorherigen Aufgabe.

\subsection{elektrisches Feld}

Für das elektrische Feld bilde ich den Gradienten:
\[
	\vec E
	= \frac{1}{4 \pi \varepsilon_0} p \vnabla \frac{z}{r^3}
	= \frac{1}{4 \pi \varepsilon_0} p \del{ \underbrace{\frac{1}{r^3} \begin{pmatrix}
		0 \\ 0 \\ 1
\end{pmatrix}}_{1/r^3 \nabla z} - \underbrace{\frac{3z}{r^4}}_{-z \nabla 1/r^3} }
\]

Das Feld ist in Abbildung \ref{plot} gezeigt. Die Äquipotenzlinien fehlen noch, sollten allerdings Hyperboloide sein.

\begin{figure}[ht]
	\centering
	\includegraphics[height=0.3\textheight]{Vektorplot.pdf}
	\caption{elektrisches Feld}
	\label{plot}
\end{figure}

%%%%%%%%%%%%%%%%%%%%%%%%%%%%%%%%%%%%%%%%%%%%%%%%%%%%%%%%%%%%%%%%%%%%%%%%%%%%%%%
%                 Trennung der Variablen in Polarkoordinaten                  %
%%%%%%%%%%%%%%%%%%%%%%%%%%%%%%%%%%%%%%%%%%%%%%%%%%%%%%%%%%%%%%%%%%%%%%%%%%%%%%%

\section{Trennung der Variablen in Polarkoordinaten}

\subsection{Lösung der Laplacegleichung}

\begin{footnotesize}
	Da das Spiel „finde die Zahl“ inzwischen langweilig ist, schlage ich als
	neues Spiel „finde den lateinischen Buchstaben“ für diese Aufgabe vor.
\end{footnotesize}

Da $\phi$ und $\varphi$ eigentlich der gleiche Buchstabe ist, benutze ich für
das Potential $\Phi$.

Als Ansatz für das Potential benutze ich $\Phi\del{\rho, \phi} = V_0 +
\Phi_\rho(\rho) \Phi_\phi(\phi)$. Dies setze ich in die Laplacegleichung, die
in Polarkoordinaten gegeben ist, ein:
\[
	\frac 1\rho \dpd{}\rho \del{\rho \dpd{\Phi_\rho}\rho \Phi_\phi} + \frac
	1{\rho^2} \dpd[2]{\Phi_\phi}\phi \Phi_\rho = 0
\]

Ich trenne die Variablen:
\[
	\frac \rho{ \Phi_\rho } \dpd{}\rho \del{\rho \dpd{\Phi_\rho}\rho} +  \frac
	1{ \Phi_\phi } \dpd[2]{\Phi_\phi}\phi  = 0
\]

Beide Seiten hängen von unterschiedlichen Variablen ab, somit müssen sie beide
gleich einer Konstanten, $\beta^2$, sein. Das Quadrat ist willkürlich und wird
erst später praktisch. Somit ergeben sich zwei Differentialgleichungen:
\[
	\rho^2 \dpd[2]{\Phi_\rho}\rho + \rho \dpd{\Phi_\rho}\rho - \beta^2 \Phi_\rho = 0
	\quad\wedge\quad
	-\beta^2 \Phi_\phi = \dpd[2]{\Phi_\phi}\phi
\]

Die Lösung für die erste Gleichung habe ich nicht selbst finden können, sie
stammt aus \cite[Seite~92]{jackson-klassische_elektrodynamik}. Die zweite
Gleichung lässt sich durch Kosinus und Sinus lösen (und jede Linearkombination
davon). Dabei stellt die Funktion $\mathcal L[M]$ die lineare Hülle der Menge
$M$ dar.
\[
	\Phi_\rho \in \mathcal L \sbr{\set{\rho^\beta, \rho^{-\beta}}}
	\quad\wedge\quad
	\Phi_\phi \in \mathcal L \sbr{\set{\cos\del{\beta \phi}, \sin\del{\beta \phi}}}
\]

Die Randbedingungen erfordern, dass $\Phi\del{\rho, 0} = 0$ und $\Phi\del{\rho,
\alpha} = 0$ sind. Daraus folgt, dass keine Kosinusterme in $\Phi_\phi$
vorkommen. Außerdem muss $\sin\del{\beta \phi} = 0$ gelten, woraus folgt:
$\beta = n \pi / \alpha$. Da $\Phi$ bei $\rho = 0$ keine Singularität haben
soll, folgt auch noch, dass keine negativen Potenzen von $\rho$ vorkommen
können. Die $\gamma_n$ sind die Koeffizienten für die Linearkombination. Die
allgemeine Lösung dann so aus:
\[
	\Phi\del{\rho, \phi} = V_0 + \sum_{n = 1}^\infty \gamma_n \rho^{n \pi /
	\alpha} \sin\del{\frac{n\pi}{\alpha} \phi}
\]

\subsection{Näherung und elektrisches Feld $\vec E$}

Für kleines $\rho$ ist nur der erste Term in der Potenzreihe maßgebend. Somit ist das Potential näherungs\-weise:
\[
	\Phi\del{\rho, \phi} \approx V_0 + \gamma_1 \rho^{\pi/\alpha} \sin\del{\frac\pi\beta\phi}
\]

Das elektrische Feld ist die negative Ableitung des Potentials. Diese sind mit den entsprechenden Formeln für Polarkoordinaten:
\[
	E_\rho = - \dpd\Phi\rho = - \gamma_1 \frac\pi\beta \rho^{\pi/\beta - 1} \sin{\frac\pi\beta\phi}
	, \quad
	E_\phi = - \frac 1\rho \dpd\Phi\phi = - \gamma_1 \frac\pi\beta \rho^{\pi/\beta - 1} \cos{\frac\pi\beta\phi}
\]

Somit ist das Feld in Polarkoordinaten:
\[
	\vec E(\rho, \phi) = \vec e_\rho E_\rho  + \vec e_\phi E_\phi
\]

Die Oberflächenladungsdichte ist gegeben durch $\sigma = \varepsilon_0 E$.
Diese ist aufgrund der Azimutalsymmetrie auf beiden Platten gleich, ich gebe
die Ladungsdichte für $\phi = 0$ an. Der Vektor $\vec \nu$ ist der
Normalenvektor auf die Platte.
\[
	\sigma(\rho)
	= \varepsilon_0 \inner{\vec \nu}{\vec E(\rho, 0)}
	= \varepsilon_0 \inner{\vec e_\phi}{\vec e_\rho E_\rho(\rho, 0)  + \vec e_\phi E_\phi(\rho, 0)}
	= - \varepsilon_0 \gamma_1 \frac\pi\beta \rho^{\pi/\beta - 1}
\]

%%%%%%%%%%%%%%%%%%%%%%%%%%%%%%%%%%%%%%%%%%%%%%%%%%%%%%%%%%%%%%%%%%%%%%%%%%%%%%%
%                           Kugelflächenfunktionen                           %
%%%%%%%%%%%%%%%%%%%%%%%%%%%%%%%%%%%%%%%%%%%%%%%%%%%%%%%%%%%%%%%%%%%%%%%%%%%%%%%

\section{Kugelflächenfunktionen}

Diese Aufgaben lasse ich aus.

%%%%%%%%%%%%%%%%%%%%%%%%%%%%%%%%%%%%%%%%%%%%%%%%%%%%%%%%%%%%%%%%%%%%%%%%%%%%%%%
%                              Spiegelladungen 2                              %
%%%%%%%%%%%%%%%%%%%%%%%%%%%%%%%%%%%%%%%%%%%%%%%%%%%%%%%%%%%%%%%%%%%%%%%%%%%%%%%

\section{Spiegelladungen 2}

Aus Symmetriegründen muss die Spiegelladung auf der Verbindungslinie vom
Koordinatenursprung und Ladung liegen. Die Ladung sei $q'$ groß und an der
Position $\vec r_{q'}$.

Ich habe mich bei der Bearbeitung an
\cite[Seite~70]{jackson-klassische_elektrodynamik} gehalten.

\subsection{Potential}

Das Potential $\Phi$ ist eine Superposition von Punktladung und Spiegelladung:
\[
	\Phi\del{\vec r} = \frac{1}{4 \pi \varepsilon_0} \del{
		\frac{q}{\abs{\vec r - \vec r_q}}
		+
		\frac{q'}{\abs{\vec r - \vec r_{q'}}}
	}
\]

Da die beiden Ladungen auf einer Linie liege, kann ich die Position als Vielfaches eines Normalenvektors $\vec \nu$ schreiben:
\[
	\Phi\del{r} = \frac{1}{4 \pi \varepsilon_0} \del{
		\frac{q}{\abs{\vec \nu r - \vec \nu r_q}}
		+
		\frac{q'}{\abs{\vec \nu r - \vec \nu r_{q'}}}
	}
\]

An der Stelle $r = R$ soll das Potential 0 sein. Somit muss gelten:
\[
	\Phi\del{R} = \frac{1}{4 \pi \varepsilon_0} \del{
		\frac{q}{\abs{\vec \nu R - \vec \nu r_q}}
		+
		\frac{q'}{\abs{\vec \nu R - \vec \nu r_{q'}}}
	}
	\overset != 0
\]

Jetzt klammere ich im Nenner aus:
\[
	\Phi\del{R} = \frac{1}{4 \pi \varepsilon_0} \del{
		\frac{q}{R \abs{\vec \nu - \vec \nu \frac{r_q}R}}
		+
		\frac{q'}{r_{q'} \abs{\vec \nu \frac{ R}{r_{q'}} - \vec \nu}}
	}
	\overset != 0
\]

Der Ausdruck ist genau dann gleich 0, wenn folgende Bedingungen erfüllt sind:
\[
	\frac qR = - \frac{q'}{r_{q'}}
	\quad\wedge\quad
	\frac{r_1}R = \frac R{r_{1'}}
\]

Dies kann ich umformen zu:
\[
	q' = - \frac R{r_q} q
	\quad\wedge\quad
	r_{q'} = \frac{R^2}{r_q}
\]

Das setze ich in das Potential ein und erhalte:
\[
	\Phi\del{\vec r} = \frac{q}{4 \pi \varepsilon_0} \del{
		\frac{1}{\abs{\vec r - \vec r_q}}
		-
		\frac{1}{\abs{\vec r \frac{r_q}R - R \vec r_{q}}}
	}
\]

\subsection{Flächenladungsdichte}

Die Flächenladungsdichte ist, mit $\gamma$ als Winkel zwischen $\vec r$ und $\vec r_q$:
\[
	\sigma = \varepsilon_0 \eval{\dpd\Phi{\vec r}}_{r=R} = - \frac{q}{4 \pi R^2} \frac R{r_q} \frac{1 - \frac{R^2}{{r_q}^2}}{\del{1 + \frac{R^2}{{r_q}^2} - R \frac R{r_q} \cos\del\gamma}^{3/2}}
\]

Die Ladung auf der Oberfläche muss gleich der Spiegelladung und das Negative
der realen Ladung sein, da der gesamte Raum ohne das Innere der Kugel durch den
Leiter begrenzt wird, der feldfrei sein muss. Somit muss außerhalb der Kugel
total keine Ladung sein.

\subsection{Kraft}

Die Kraft auf die Ladung könnte durch Integration über die Ladungsverteilung ermittelt werden. Ich wähle hier den einfacheren Weg, in dem ich die Kraft auf die Spiegelladung berechne. Diese ist:
\[
	\vec F
	= \frac{1}{4 \pi \varepsilon_0} \frac{q^2}{d^3} \vec d
\]


Dann setze ich noch die in den vorherigen Aufgaben gefundene Relationen für die Position und Ladung der Spiegelladung ein. Damit erhalte ich:
\[
	\vec F
	= \frac{1}{4 \pi \varepsilon_0} \frac{q^2}{{r_q}^2} \del{\frac a{r_q}}^3 \del{1-\frac{a^2}{{r_q}^2}}^{-2} \hat{\vec r}_q
\]

\subsection{isolierte, leitende Hohlkugel}

Laut \cite[Seite~75]{jackson-klassische_elektrodynamik} kann ich die Ladung $Q$
auf der Kugeloberfläche als lineare Superposition von einer Ladung $q'$, die
die äußere Ladung $q$ ausgleicht und einer Restladung $Q - q'$ ansehen. Dabei
verteilt sich die Restladung homogen, weil die Ladung $q'$ bereits das externe
Feld der Ladung $q$ ausgleicht.

Somit kommen als zu den Potentialen und Kräften der vorherigen Aufgabenteilen
folgende Teile hinzu. Dabei kann ich die Restladung im Koordinatenursprung
zentriert annehmen.
\[
	\Phi_{Q + q'}\del{\vec x}
	= \frac{1}{4 \pi \varepsilon_0} \frac{Q+\frac R{r_q} q}{x}
\]

Die Kraft bekommt ebenfalls einen weiteren Term:
\[
	\vec F_{Q + q'}\del{\vec x}
	= \frac{1}{4 \pi \varepsilon_0} \frac{\del{Q+\frac R{r_q} q}q}{x^3} \vec x
\]

Ein Ladungsüberschuss auf der Außenseite wird diese nicht verlassen und auf die
Innenseite wandern. Da innerhalb des Leiters kein Feld herrschen kann, muss
nach dem Gesetz von Gauß das Innere ladungsfrei sein. Daher kann die Ladung
nicht nach innen wandern. Von der Kugel könnte die Ladung bei ausreichend hohen
elektrischen Feldern und realen Leitern allerdings in einem Blitz entkommen.

\bibliography{../../zentrale_BibTeX/Central}
\bibliographystyle{plain}

\end{document}

% vim: spell spelllang=de
