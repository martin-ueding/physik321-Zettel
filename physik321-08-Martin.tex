% Copyright © 2012 Martin Ueding <dev@martin-ueding.de>
%
\documentclass[11pt, ngerman, fleqn]{article}

\usepackage[a4paper, left=3cm, right=2cm, top=2cm, bottom=2cm]{geometry}
\usepackage[activate]{pdfcprot}
\usepackage[cdot, squaren]{SIunits}
\usepackage[iso]{isodate}
\usepackage[parfill]{parskip}
\usepackage[scaled]{beramono}
\usepackage[T1]{fontenc}
\usepackage[utf8]{inputenc}
\usepackage{amsmath}
\usepackage{amsthm}
\usepackage{babel}
\usepackage{color}
\usepackage{commath}
\usepackage{fancyhdr}
\usepackage{graphicx}
\usepackage{hyperref}
\usepackage{lastpage}
\usepackage{minted}
\usepackage{setspace}
\usepackage{tikz}

\usepackage[charter, greekuppercase=italicized]{mathdesign}

\definecolor{darkblue}{rgb}{0,0,.5}
\definecolor{darkgreen}{rgb}{0,.5,0}

\hypersetup{
	breaklinks=false,
	citecolor=darkgreen,
	colorlinks=true,
	linkcolor=black,
	menucolor=black,
	urlcolor=darkblue,
}

\setlength{\columnsep}{2cm}

\DeclareMathOperator{\arcsinh}{arsinh}
\DeclareMathOperator{\arsinh}{arsinh}
\DeclareMathOperator{\asinh}{arsinh}
\DeclareMathOperator{\card}{card}
\DeclareMathOperator{\diam}{diam}

\newcommand{\curl}{\vnabla \times}
\newcommand{\dalambert}{\mathop{{}\Box}\nolimits}
\newcommand{\divergence}[1]{\inner{\vnabla}{#1}}
\newcommand{\ee}{\mathrm e}
\newcommand{\emesswert}{\del{\messwert \pm \messwert}}
\newcommand{\ev}{\hat{\vec e}}
\newcommand{\e}[1]{\cdot 10^{#1}}
\newcommand{\fehlt}{\textcolor{red}{Hier fehlen noch Inhalte.}}
\newcommand{\half}{\frac 12}
\newcommand{\ii}{\mathrm i}
\newcommand{\inner}[2]{\left\langle #1, #2 \right\rangle}
\newcommand{\laplace}{\mathop{{}\bigtriangleup}\nolimits}
\newcommand{\messwert}{\textcolor{blue}{\square}}
\newcommand{\punkte}{\textcolor{white}{xxxxx}}
\newcommand{\tens}[1]{\boldsymbol{\mathsf{#1}}}
\newcommand{\vnabla}{\vec \nabla}
\renewcommand{\vec}[1]{\boldsymbol{#1}}

\newcommand{\themodul}{physik321}
\newcommand{\thegruppe}{Gruppe 8 -- Julia Volmer}
\newcommand{\theuebung}{8}

\pagestyle{fancy}

\fancyfoot[C]{\footnotesize{\thegruppe}}
\fancyfoot[L]{\footnotesize{Martin Ueding, Simon Schlepphorst}}
\fancyfoot[R]{\footnotesize{Seite \thepage\ / \pageref{LastPage}}}
\fancyhead[L]{\themodul{} -- Übung \theuebung}

\def\thesection{H \theuebung.\arabic{section}}
\def\thesubsubsection{\thesubsection\alph{section}}

\title{\themodul{} -- Übung \theuebung \\ \vspace{0.5cm} \large{\thegruppe}}

\author{
	Martin Ueding \\ \small{\href{mailto:mu@uni-bonn.de}{mu@uni-bonn.de}}
	\and
	Simon Schlepphorst \\ \small{\href{mailto:s2@uni-bonn.de}{s2@uni-bonn.de}}
}

\begin{document}

\maketitle

\begin{table}[h]
	\centering
	\begin{tabular}{l|c|c|c|c}
		Aufgabe & \ref 1 & \ref 2 & \ref 3 & $\sum$   \\
		\hline
		Punkte & \punkte / 10 & \punkte / 20 & \punkte / 10 & \punkte / 30
	\end{tabular}
\end{table}


%\bibliography{../../zentrale_BibTeX/Central}
%\bibliographystyle{plain}

%%%%%%%%%%%%%%%%%%%%%%%%%%%%%%%%%%%%%%%%%%%%%%%%%%%%%%%%%%%%%%%%%%%%%%%%%%%%%%%
%               potentielle Energie des statischen Magnetfeldes               %
%%%%%%%%%%%%%%%%%%%%%%%%%%%%%%%%%%%%%%%%%%%%%%%%%%%%%%%%%%%%%%%%%%%%%%%%%%%%%%%

\section{potentielle Energie des statischen Magnetfeldes}
\label 1

Wir beginnen mit:
\begin{align*}
	P &= \int \dif V \, \inner{\vec E}{\vec j} \\
	\intertext{Wir setzen ein, dass $\curl \vec H - \vec j = 0$ ist.}
	P &= \int \dif V \, \inner{\vec E}{\curl \vec H} \\
	\intertext{Nun benutzen wir die gegebene Identität.}
	P &= \int \dif V \, \del{\inner{\vnabla}{\vec E \times \vec H} \inner{\vec H}{\curl \vec E}} \\
	\intertext{Die Rotation des elektrischen Feldes ist gerade $- \dot{\vec B}$.}
	P &= \int \dif V \, \del{\inner{\vnabla}{\vec E \times \vec H} \inner{\vec H}{\dot{\vec B}}} \\
	\intertext{Für den ersten Summanden wenden wir den Satz von Gauß an. Im zweiten Summanden wenden wir an, dass $\frac{1}{\mu} H = B$ ist.}
	P &= \int\limits_{\partial V} \dif A \, \inner{\nu}{\vec E \times \vec H} - \frac{1}{2\mu} \int \dif V \, \dpd{}t B^2 \\
	\intertext{Das Oberflächenintegral ist über den Poyntingvektor, also misst es die Leistung, die durch Wellen nach Außen getragen werden. Falls das Magnetfeld also keine Leistung nach Außen trägt, kann dieses Integral vernachlässigt werden.}
	P &= - \frac{1}{2\mu} \int \dif V \, \dpd{}t B^2 \\
	\intertext{Wir integrieren nach der Zeit und erhalten die Energie im Magnetfeld.}
	U &= - \frac{1}{2\mu} \int \dif V \int_0^t \dif{t'} \, \dpd{}{t'} B^2\del{t'} \\
	U &= - \frac{1}{2\mu} \int \dif V \, B^2\del{t}
\end{align*}

Dies ist die gesuchte Relation.

%%%%%%%%%%%%%%%%%%%%%%%%%%%%%%%%%%%%%%%%%%%%%%%%%%%%%%%%%%%%%%%%%%%%%%%%%%%%%%%
%                                 Regenbogen                                  %
%%%%%%%%%%%%%%%%%%%%%%%%%%%%%%%%%%%%%%%%%%%%%%%%%%%%%%%%%%%%%%%%%%%%%%%%%%%%%%%

\section{Regenbogen}
\label 2

\fehlt

%%%%%%%%%%%%%%%%%%%%%%%%%%%%%%%%%%%%%%%%%%%%%%%%%%%%%%%%%%%%%%%%%%%%%%%%%%%%%%%
%                  potentielle Energie eines Ionenkristalls                   %
%%%%%%%%%%%%%%%%%%%%%%%%%%%%%%%%%%%%%%%%%%%%%%%%%%%%%%%%%%%%%%%%%%%%%%%%%%%%%%%

\section{potentielle Energie eines Ionenkristalls}
\label 3

Das Durchzählen der Punkte ist relativ simpel realisiert, wir zählen alle
Punkte
\[
	\set{(x, y, z)\colon \min\set{x, y, z} \leq a}
\] durch. Dabei ist $a = 0, 1, 2, \ldots$, wobei der Punkt $(0, 0, 0)$
ausgeschlossen worden ist. Das Vorzeichen der Ladung an der Stelle $(x, y, z)$
haben wir durch $(-1)^x (-1)^y (-1)^z$ bestimmt. Somit ist der Ursprung, auf
den sich die potentielle Energie bezieht, in einer positiven Ladung zentriert.

Unser Programm für diese Aufgabe haben wir in Python 3 implementiert. Es kann
unter \url{http://uni-bonn.de/~s6mauedi/physik321/} oder
\url{https://github.com/martin-ueding/physik321-08} (\texttt{H3.py})
heruntergeladen werden.

\inputminted[fontsize=\small, frame=lines, linenos, mathescape]{python}{H3.py}

Die Ausgabe des Programms:

\inputminted[fontsize=\small, frame=lines]{text}{H3.txt}

Der Algorithmus hängt leider $\mathcal O\del{n^3}$ von der Eingabe ab, so dass
es ab $n = 60$ mehr als eine halbe Minute dauert. Das Ergebnis ändert sich
allerdings nur noch wenig, für $n = 64$ haben wir $U =
\unit{-1.53015901663\e{-28}}\volt$ sowie für $n = 128$ haben wir $U =
\unit{-1.540450919\e{-28}}\volt$ erhalten. Für eine erste Abschätzung sollte
dies reichen.

\end{document}

% vim: spell spelllang=de
