% Copyright © 2012 Martin Ueding <dev@martin-ueding.de>
%
\documentclass[11pt, ngerman, fleqn]{article}

\usepackage[a4paper, left=3cm, right=2cm, top=2cm, bottom=2cm]{geometry}
\usepackage[activate]{pdfcprot}
\usepackage[cdot, squaren]{SIunits}
\usepackage[iso]{isodate}
\usepackage[parfill]{parskip}
\usepackage[T1]{fontenc}
\usepackage[utf8]{inputenc}
\usepackage{amsmath}
\usepackage{amsthm}
\usepackage{babel}
\usepackage{color}
\usepackage{commath}
\usepackage{fancyhdr}
\usepackage{graphicx}
\usepackage{hyperref}
\usepackage{lastpage}
\usepackage{setspace}
\usepackage{tikz}

\usepackage[charter, greekuppercase=italicized]{mathdesign}

\definecolor{darkblue}{rgb}{0,0,.5}
\definecolor{darkgreen}{rgb}{0,.5,0}

\hypersetup{
	breaklinks=false,
	citecolor=darkgreen,
	colorlinks=true,
	linkcolor=black,
	menucolor=black,
	urlcolor=darkblue,
}

\setlength{\columnsep}{2cm}

\DeclareMathOperator{\arcsinh}{arsinh}
\DeclareMathOperator{\arsinh}{arsinh}
\DeclareMathOperator{\asinh}{arsinh}
\DeclareMathOperator{\card}{card}
\DeclareMathOperator{\diam}{diam}

\newcommand{\dalambert}{\mathop{{}\Box}\nolimits}
\newcommand{\divergence}[1]{\inner{\vnabla}{#1}}
\newcommand{\ee}{\mathrm e}
\newcommand{\emesswert}{\del{\messwert \pm \messwert}}
\newcommand{\ev}{\hat{\vec e}}
\newcommand{\e}[1]{\cdot 10^{#1}}
\newcommand{\fehlt}{\textcolor{red}{Hier fehlen noch Inhalte.}}
\newcommand{\half}{\frac 12}
\newcommand{\ii}{\mathrm i}
\newcommand{\inner}[2]{\left\langle #1, #2 \right\rangle}
\newcommand{\laplace}{\mathop{{}\Deltaup}\nolimits}
\newcommand{\messwert}{\textcolor{blue}{\square}}
\newcommand{\punkte}{\textcolor{white}{xxxxx}}
\newcommand{\tens}[1]{\boldsymbol{\mathsf{#1}}}
\newcommand{\vnabla}{\vec \nabla}
\renewcommand{\vec}[1]{\boldsymbol{#1}}

\newcommand{\themodul}{physik321}
\newcommand{\thegruppe}{Gruppe 8 -- Julia Volmer}
\newcommand{\theuebung}{5}

\pagestyle{fancy}

\fancyfoot[C]{\footnotesize{\thegruppe}}
\fancyfoot[L]{\footnotesize{Martin Ueding, Simon Schlepphorst}}
\fancyfoot[R]{\footnotesize{Seite \thepage\ / \pageref{LastPage}}}
\fancyhead[L]{\themodul{} -- Übung \theuebung}

\def\thesection{H \theuebung.\arabic{section}}
\def\thesubsubsection{\thesubsection\alph{section}}

\title{\themodul{} -- Übung \theuebung \\ \vspace{0.5cm} \large{\thegruppe}}

\author{
	Martin Ueding \\ \small{\href{mailto:mu@uni-bonn.de}{mu@uni-bonn.de}}
	\and
	Simon Schlepphorst \\ \small{\href{mailto:s2@uni-bonn.de}{s2@uni-bonn.de}}
}

\begin{document}

\maketitle

\begin{table}[h]
	\centering
	\begin{tabular}{l|c|c|c|c}
		Aufgabe & \ref 1 & \ref 2 & \ref 3 & $\sum$   \\
		\hline
		Punkte & \punkte / 20 & \punkte / 10 & \punkte / 10 & \punkte / 40
	\end{tabular}
\end{table}

%%%%%%%%%%%%%%%%%%%%%%%%%%%%%%%%%%%%%%%%%%%%%%%%%%%%%%%%%%%%%%%%%%%%%%%%%%%%%%%
%                         Magnetfeld an Grenzflächen                         %
%%%%%%%%%%%%%%%%%%%%%%%%%%%%%%%%%%%%%%%%%%%%%%%%%%%%%%%%%%%%%%%%%%%%%%%%%%%%%%%

\section{Magnetfeld an Grenzflächen}
\label 1

\subsection{Gradient}

Es gibt keine Stromdichte, somit gilt überall $\vec j = 0$. Daraus folgt für
$\vec H$:
\[
	\vnabla \times \vec H = \vec j = 0
\]

Somit ist $\vec H$ rotationsfrei und kann als Gradient geschrieben werden.

Die magnetische Flussdichte setzt sich aus der Erregung und Magnetisierung
zusammen:
\[
	\vec B = \mu_0 \del{\vec H + \vec M}
\]

Da allerdings auch $\divergence{\vec B} = 0$ gelten muss, gilt
$\divergence{\vec H + \vec M} = 0$.

\fehlt

Mit dem Poissonintegral können wir das skalare Potential $\varphi_m$ bestimmen
\cite[Seite 200]{nolting-elektrodynamik}:
\begin{align*}
	\varphi_m\del{\vec r}
	&= - \frac{1}{4 \pi} \int \dif^3 r' \, \frac{\divergence{\vec M\del{\vec r'}}}{\abs{\vec r - \vec r'}} \\
	&= - \frac{M_0}{4 \pi} \dod{}z \int \dif^3 r' \, \frac{1}{\abs{\vec r - \vec r'}} \\
	&= - \frac{M_0}{4 \pi} \dod{}z \frac{4 \pi R^3}{3r} \\
	&= \frac{M_0 R^3 \cos\del{\angle\del{\vec r, \ev_z}}}{3r}
\end{align*}

Das magnetische Moment der Kugel ist, da das Feld homogen ist:
\begin{equation}
	\label{eq:1-m}
	\vec m = \int \dif^3 r' \, \vec M \del{\vec r'} = \frac 43 \pi R^3 M_0 \ev_z
\end{equation}

Zusammen erhalten wir:
\[
	\varphi_m \del{\vec r}
	= \frac{1}{4\pi} \frac{\inner{\vec m}{\vec r}}{r^3}
\]

\subsection{Magnetfeld innerhalb und außerhalb}

\fehlt

\subsection{Oberflächenstromdichte}

Es soll erklärt werden, warum die Oberflächenstromdichte folgende Form haben
muss:
\[
	\vec j = \alpha\del\theta \delta\del{r-R} \ev_\phi
\]

Damit die Magnetisierung parallel zur $z$-Achse ist, muss der Strom ein
Wirbelfeld um die $z$-Achse sein. Die Richtung von $\phi$ sind konzentrische
Kreise um die $z$-Achse, womit der Strom in $\ev_\phi$ Richtung fließen
muss. Daher das $\ev_\phi$.

Das magnetische Moment ist definiert als:
\[
	m := \half \int \dif^3 r \, \del{\vec r \times \vec j\del{\vec r}}
\]

Dies muss gerade \eqref{eq:1-m} sein. Somit erhalten wir als Gleichung für $\vec j$:
\[
	\half \int \dif^3 r \del{\vec r \times \vec j\del{\vec r}}
	= \frac 43 \pi R^3 M_0 \ev_z
\]

Wir können $\vec r \times \ev_\phi = - r \ev_\theta$ in das Integral für die Stromdichte einsetzen und erhalten:
\begin{align*}
	- \half \int \dif^3 r \alpha(\theta) \delta(r-R) r \ev_\theta
	&= - \half \int_0^\pi \dif \theta \int_0^{2 \pi} \dif \phi \int_0^R \dif r \, r^2 \sin\del\theta r \alpha(\theta) \delta(r-R) r \ev_\theta \\
	&= - \half \int_0^\pi \dif \theta \int_0^{2 \pi} \dif \phi \, R^3 \sin\del\theta \alpha(\theta) \ev_\theta \\
	\intertext{An dieser Stelle setzen wir $\ev_\theta$ in kartesischen Koordinaten ein, damit wir später auch ein $\ev_z$ erhalten können. Dieses ist: $\ev_\theta = \del{\cos\del\phi \cos\del\theta, \sin\del\phi \cos\del\theta, - \sin\del\theta}^T$.}
	&= - \half \int_0^\pi \dif \theta \int_0^{2 \pi} \dif \phi \, R^3 \sin\del\theta \alpha(\theta) \begin{pmatrix}
		\cos\del\phi \cos\del\theta \\ \sin\del\phi \cos\del\theta \\ - \sin\del\theta
	\end{pmatrix} \\
	\intertext{Jetzt nutzen wir aus, dass $\int_0^{2\pi} \dif \phi \, \cos\del\phi = 0$, sowie $\int_0^{2\pi} \dif \phi \, \sin\del\phi = 0$ und $\int_0^{2\pi} \dif \phi = 2\pi$.}
	&= \pi R^3 \ev_z \int_0^\pi \dif \theta \, \sin\del\theta \alpha(\theta)
\end{align*}

Dies sollte jetzt gerade $\frac 43 \pi R^3 M_0 \ev_z$ sein.
\begin{align*}
	\pi R^3 \ev_z \int_0^\pi \dif \theta \, \sin\del\theta \alpha(\theta) &= \frac 43 \pi R^3 M_0 \ev_z \\
	\int_0^\pi \dif \theta \, \sin\del\theta \alpha(\theta) &= \frac 43 M_0 \\
	\int_0^\pi \dif \theta \, \sin\del\theta \alpha(\theta) &= \frac 1\pi \int_0^\pi \dif \theta \, \frac 43 M_0 \\
	\alpha\del\theta &= \frac 43 \frac 1\pi M_0 \csc^2\del\theta
\end{align*}

Somit ist die Stromverteilung:
\[
	\vec j = \frac 4{3\pi} M_0 \csc^2\del\theta \delta\del{r-R} \ev_\phi
\]

%%%%%%%%%%%%%%%%%%%%%%%%%%%%%%%%%%%%%%%%%%%%%%%%%%%%%%%%%%%%%%%%%%%%%%%%%%%%%%%
%                   Konzentrische Zylinder mit Dielektrikum                   %
%%%%%%%%%%%%%%%%%%%%%%%%%%%%%%%%%%%%%%%%%%%%%%%%%%%%%%%%%%%%%%%%%%%%%%%%%%%%%%%

\section{Konzentrische Zylinder mit Dielektrikum}
\label 2

\fehlt

%%%%%%%%%%%%%%%%%%%%%%%%%%%%%%%%%%%%%%%%%%%%%%%%%%%%%%%%%%%%%%%%%%%%%%%%%%%%%%%
%                 Greensfunktionen in $d = 2, 1$ Dimensionen                  %
%%%%%%%%%%%%%%%%%%%%%%%%%%%%%%%%%%%%%%%%%%%%%%%%%%%%%%%%%%%%%%%%%%%%%%%%%%%%%%%

\section{Greensfunktionen in $d = 2, 1$ Dimensionen}
\label 3

\fehlt

\bibliography{../../zentrale_BibTeX/Central}
\bibliographystyle{plain}

\end{document}

% vim: spell spelllang=de
