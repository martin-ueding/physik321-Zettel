% Copyright © 2012 Martin Ueding <dev@martin-ueding.de>
%
\documentclass[11pt, ngerman, fleqn]{article}

\usepackage[a4paper, left=3cm, right=2cm, top=2cm, bottom=2cm]{geometry}
\usepackage[activate]{pdfcprot}
\usepackage[cdot, squaren]{SIunits}
\usepackage[iso]{isodate}
\usepackage[parfill]{parskip}
\usepackage[T1]{fontenc}
\usepackage[utf8]{inputenc}
\usepackage{amsmath}
\usepackage{amsthm}
\usepackage{babel}
\usepackage{color}
\usepackage{commath}
\usepackage{fancyhdr}
\usepackage{graphicx}
\usepackage{hyperref}
\usepackage{lastpage}
\usepackage{setspace}
\usepackage{tikz}

\usepackage[charter, greekuppercase=italicized]{mathdesign}

\definecolor{darkblue}{rgb}{0,0,.5}
\definecolor{darkgreen}{rgb}{0,.5,0}

\hypersetup{
	breaklinks=false,
	citecolor=darkgreen,
	colorlinks=true,
	linkcolor=black,
	menucolor=black,
	urlcolor=darkblue,
}

\setlength{\columnsep}{2cm}

\DeclareMathOperator{\arcsinh}{arsinh}
\DeclareMathOperator{\arsinh}{arsinh}
\DeclareMathOperator{\asinh}{arsinh}
\DeclareMathOperator{\card}{card}
\DeclareMathOperator{\diam}{diam}

\newcommand{\dalambert}{\mathop{{}\Box}\nolimits}
\newcommand{\divergence}[1]{\inner{\vnabla}{#1}}
\newcommand{\ee}{\mathrm e}
\newcommand{\emesswert}{\del{\messwert \pm \messwert}}
\newcommand{\ev}{\hat{\vec e}}
\newcommand{\e}[1]{\cdot 10^{#1}}
\newcommand{\fehlt}{\textcolor{red}{Hier fehlen noch Inhalte.}}
\newcommand{\half}{\frac 12}
\newcommand{\ii}{\mathrm i}
\newcommand{\inner}[2]{\left\langle #1, #2 \right\rangle}
\newcommand{\laplace}{\mathop{{}\Deltaup}\nolimits}
\newcommand{\messwert}{\textcolor{blue}{\square}}
\newcommand{\punkte}{\textcolor{white}{xxxxx}}
\newcommand{\tens}[1]{\boldsymbol{\mathsf{#1}}}
\newcommand{\vnabla}{\vec \nabla}
\renewcommand{\vec}[1]{\boldsymbol{#1}}

\newcommand{\themodul}{physik321}
\newcommand{\thegruppe}{Gruppe 8 -- Julia Volmer}
\newcommand{\theuebung}{6}

\pagestyle{fancy}

\fancyfoot[C]{\footnotesize{\thegruppe}}
\fancyfoot[L]{\footnotesize{Martin Ueding, Simon Schlepphorst}}
\fancyfoot[R]{\footnotesize{Seite \thepage\ / \pageref{LastPage}}}
\fancyhead[L]{\themodul{} -- Übung \theuebung}

\def\thesection{H \theuebung.\arabic{section}}
\def\thesubsubsection{\thesubsection\alph{section}}

\title{\themodul{} -- Übung \theuebung \\ \vspace{0.5cm} \large{\thegruppe}}

\author{
	Martin Ueding \\ \small{\href{mailto:mu@uni-bonn.de}{mu@uni-bonn.de}}
	\and
	Simon Schlepphorst \\ \small{\href{mailto:s2@uni-bonn.de}{s2@uni-bonn.de}}
}

\begin{document}

\maketitle

\begin{table}[h]
	\centering
	\begin{tabular}{l|c|c|c|c|c}
		Aufgabe & \ref 1 & \ref 2 & \ref 3 & \ref 4 & $\sum$   \\
		\hline
		Punkte & \punkte / 10 & \punkte / 10 & \punkte / 20 & \punkte / 20 & \punkte / 50
	\end{tabular}
\end{table}

%%%%%%%%%%%%%%%%%%%%%%%%%%%%%%%%%%%%%%%%%%%%%%%%%%%%%%%%%%%%%%%%%%%%%%%%%%%%%%%
%                                 Lorenzkraft                                 %
%%%%%%%%%%%%%%%%%%%%%%%%%%%%%%%%%%%%%%%%%%%%%%%%%%%%%%%%%%%%%%%%%%%%%%%%%%%%%%%

\section{Lorenzkraft}
\label 1

\subsection{qualitative Teilchenbahn und Bewegungsgleichungen}

Das Teilchen ist im konservativen $\vec E$-Feld. Dort fällt es entlang des
Feldes, wird allerdings vom $\vec B$-Feld abgelenkt und wird gegen das Feld
gelenkt. Das Teilchen kommt nur so weit, bis es die $z$-Stelle des
Anfangspunkts erreicht hat. Das ganze hatte ich in der Schule in einer Aufgabe,
dazu hatte ich ein Programm
geschrieben\footnote{\url{http://martin-ueding.de/projects/van-allen-sim-3d/}}.
Ein Bildschirmfoto des Programms ist in Abbildung \ref{fig:ebsim}.

\begin{figure}
	\centering
	\includegraphics[width=.5\textwidth]{ebsim2.png}
	\caption{Bildschirmfoto des Programms}
	\label{fig:ebsim}
\end{figure}


Das Teilchen bewegt sich auf einer Zykloiden, im Mittel senkrecht zu $\vec E$-
und $\vec B$-Feld, also entlang der $y$-Achse.

Die Kraft auf das Teilchen ist:
\[
	\vec F = q \del{\vec E + \vec v \times \vec B}
\]

In Komponenten geschrieben:
\[
	\ddot r^i = \frac qm \del{E \delta^i{}_z + \epsilon^i{}_{jk} \dot r^j B \delta^k{}_x}
\]

Daraus ergeben sich drei gekoppelte, partielle Differentialgleichungen:
\[
	\ddot x = 0
	,\quad
	\ddot y = \frac qm \dot z B
	,\quad
	\ddot z = \frac qm \del{E - \dot y B}
\]

Für $x$ gilt aufgrund der Anfangsbedingungen: $x(t) = 0$. Die Gleichungen für
$y$ und $z$ können wir jeweils nach der Zeit ableiten und in die Andere
einsetzen. So erhalten wir zwei unabhängige Differentialgleichungen höherer
Ordnung:
\[
	\dddot y = \frac{q^2}{m^2} B E - \frac{q^2}{m^2} B^2 \dot y
	,\quad
	\dddot z = - \frac{q^2}{m^2} B^2 \dot z
\]

Diese Bewegungen sind, von additiven linearen Bewegungen abgesehen, harmonische
Schwingungen.

\subsection{Lösung der Gleichungen}

Die zu $y$ gehörige homogene Gleichung können wir durch Kosinus und Sinus
lösen:
\[
	y_h(t) = c_1 \cos\del{\omega t} + c_2 \sin\del{\omega t}
\]

Dabei haben wir schon die Ersetzung $\omega := q B / m$ vorgenommen. Als
inhomogene Lösung erraten wir:
\[
	y_p(t) = \frac EB t + c_3
\]

Somit gilt für $y$:
\[
	y(t) = c_1 \cos\del{\omega t} + c_2 \sin\del{\omega t} + \frac EB t + c_3
\]

Ähnlich gehen wir für $z$ vor und erhalten:
\[
	z(t) = c_4 \cos\del{\omega t} + c_5 \sin\del{\omega t} + c_6
\]

Die komplette Lösung für den Positionsvektor $\vec r$ ist somit:
\[
	\vec r(t) = \begin{pmatrix}
		0 \\
		c_1 \\
		c_4 \\
	\end{pmatrix} \cos\del{\omega t}
	+
	\begin{pmatrix}
		0 \\
		c_2 \\
		c_5 \\
	\end{pmatrix} \sin\del{\omega t}
	+
	\begin{pmatrix}
		0 \\
		\frac EB \\
		0 \\
	\end{pmatrix} t
	+
	\begin{pmatrix}
		0 \\
		c_3 \\
		c_6 \\
	\end{pmatrix}
\]

Nun war in der Aufgabenstellung gefordert, dass $\vec r(0) = \dot{\vec r}(0) =
\vec 0$. Die Ableitung nach der Zeit ist:
\[
	\dot{\vec r}(t) = \begin{pmatrix}
		0 \\
		- \omega c_1 \\
		- \omega c_4 \\
	\end{pmatrix} \sin\del{\omega t}
	+
	\begin{pmatrix}
		0 \\
		\omega c_2 \\
		\omega c_5 \\
	\end{pmatrix} \cos\del{\omega t}
	+
	\begin{pmatrix}
		0 \\
		\frac EB \\
		0 \\
	\end{pmatrix}
\]

Daraus erhalten wir:
\[
	c_2 = - \frac{E}{\omega B}
	,\quad
	c_5 = 0
\]

Weil wir die Gleichungen zur Lösung noch einmal abgeleitet hatten, müssen wir
noch die Bedingung $\ddot{\vec r}(0) = \frac qm E \ev_z$ erfüllen. Somit erhalten wir noch:
\[
	\dot{\vec r}(0) = \begin{pmatrix}
		0 \\
		- \omega^2 c_1 \\
		- \omega^2 c_4 \\
	\end{pmatrix}
	= \frac qm E \ev_z
\]

Dies können wir umstellen und erhalten zusammen mit $c_1 + c_3 = 0$ sowie $c_4
+ c_6 = 0$ aus $\dot{\vec r}(0) = \vec 0$:
\[
	c_1 = 0
	,\quad
	c_3 = 0
	,\quad
	c_4 = - \frac{E}{\omega B}
	, \quad
	c_6 = \frac{E}{\omega B}
\]

Alles zusammen ergibt die Bahnkurve:
\[
	\vec r(t) = - \frac{E}{\omega B} \begin{pmatrix}
		0 \\
		\sin\del{\omega t} \\
		\cos\del{\omega t} \\
	\end{pmatrix}
	+
	\begin{pmatrix}
		0 \\
		\frac EB \\
		0 \\
	\end{pmatrix} t
	+
	\begin{pmatrix}
		0 \\
		0 \\
		\frac{E}{\omega B} \\
	\end{pmatrix}
\]

Die Bahnkurve ist für $t \in \intoo{0, 15}$ in Abbildung \ref{fig:zyklo}
dargestellt. Dies stimmt mit den Überlegungen und der Simulation (Abbildung
\ref{fig:ebsim}) überein.

\begin{figure}
	\centering
	\includegraphics[width=0.7\textwidth]{1-zyklo.pdf}
	\caption{Bahnkurve $\vec r(t)$ für $t \in \intoo{0, 15}$}
	\label{fig:zyklo}
\end{figure}

%%%%%%%%%%%%%%%%%%%%%%%%%%%%%%%%%%%%%%%%%%%%%%%%%%%%%%%%%%%%%%%%%%%%%%%%%%%%%%%
%                         Kraft auf elementaren Dipol                         %
%%%%%%%%%%%%%%%%%%%%%%%%%%%%%%%%%%%%%%%%%%%%%%%%%%%%%%%%%%%%%%%%%%%%%%%%%%%%%%%

\section{Kraft auf elementaren Dipol}
\label 2

\fehlt

%%%%%%%%%%%%%%%%%%%%%%%%%%%%%%%%%%%%%%%%%%%%%%%%%%%%%%%%%%%%%%%%%%%%%%%%%%%%%%%
%                              Helmholtz-Spulen                               %
%%%%%%%%%%%%%%%%%%%%%%%%%%%%%%%%%%%%%%%%%%%%%%%%%%%%%%%%%%%%%%%%%%%%%%%%%%%%%%%

\section{Helmholtz-Spulen}
\label 3

\fehlt

%%%%%%%%%%%%%%%%%%%%%%%%%%%%%%%%%%%%%%%%%%%%%%%%%%%%%%%%%%%%%%%%%%%%%%%%%%%%%%%
%               Eigenschaften ebener elektromagnetischer Wellen               %
%%%%%%%%%%%%%%%%%%%%%%%%%%%%%%%%%%%%%%%%%%%%%%%%%%%%%%%%%%%%%%%%%%%%%%%%%%%%%%%

\section{Eigenschaften ebener elektromagnetischer Wellen}
\label 4

\fehlt

%\bibliography{../../zentrale_BibTeX/Central}
%\bibliographystyle{plain}

\end{document}

% vim: spell spelllang=de
