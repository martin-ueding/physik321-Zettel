% Copyright © 2012 Martin Ueding <dev@martin-ueding.de>
%
\documentclass[11pt, ngerman]{article}

\usepackage[a4paper, left=3cm, right=2cm, top=2cm, bottom=2cm]{geometry}
\usepackage[activate]{pdfcprot}
\usepackage[cdot, squaren]{SIunits}
\usepackage[iso]{isodate}
\usepackage[parfill]{parskip}
\usepackage[T1]{fontenc}
\usepackage[utf8]{inputenc}
\usepackage{amsmath}
\usepackage{amssymb}
\usepackage{amsthm}
\usepackage{babel}
\usepackage{color}
\usepackage{commath}
\usepackage{epstopdf}
\usepackage{fancyhdr}
\usepackage{graphicx}
\usepackage{hyperref}
\usepackage{setspace}
\usepackage{tikz}

\usepackage[charter]{mathdesign}

\definecolor{darkblue}{rgb}{0,0,.5}
\definecolor{darkgreen}{rgb}{0,.5,0}

\hypersetup{
	breaklinks=false,
	citecolor=darkgreen,
	colorlinks=true,
	linkcolor=black,
	menucolor=black,
	urlcolor=darkblue,
}

\setlength{\columnsep}{2cm}

\DeclareMathOperator{\arcsinh}{arsinh}
\DeclareMathOperator{\arsinh}{arsinh}
\DeclareMathOperator{\asinh}{arsinh}
\DeclareMathOperator{\card}{card}
\DeclareMathOperator{\diam}{diam}

\newcommand{\dalambert}{\mathop{{}\Box}\nolimits}
\newcommand{\divergence}[1]{\inner{\vnabla}{#1}}
\newcommand{\ee}{\mathrm e}
\newcommand{\emesswert}{\del{\messwert \pm \messwert}}
\newcommand{\e}[1]{\cdot 10^{#1}}
\newcommand{\fehlt}{\textcolor{red}{Hier fehlen noch Inhalte.}}
\newcommand{\half}{\frac 12}
\newcommand{\ii}{\mathrm i}
\newcommand{\inner}[2]{\left\langle #1, #2 \right\rangle}
\newcommand{\laplace}{\mathop{{}\bigtriangleup}\nolimits}
\newcommand{\messwert}{\textcolor{blue}{\square}}
\newcommand{\punkte}{\textcolor{white}{xxxxx}}
\newcommand{\tens}[1]{\boldsymbol{#1}}
\newcommand{\vnabla}{\vec \nabla}
\renewcommand{\vec}[1]{\boldsymbol{#1}}

\newcommand{\themodul}{physik321}
\newcommand{\thegruppe}{Gruppe 8 -- Julia Volmer}
\newcommand{\theuebung}{2}

\pagestyle{fancy}

\fancyfoot[C]{\footnotesize{\thegruppe}}
\fancyfoot[L]{\footnotesize{Martin Ueding, Simon Schlepphorst}}
\fancyfoot[R]{\footnotesize{Seite \thepage}}
\fancyhead[L]{\themodul{} -- Übung \theuebung}

\setcounter{section}{0}

\def\thesection{H \theuebung.\arabic{section}}
\def\thesubsection{\thesection\alph{subsection}}

\title{\themodul{} -- Übung \theuebung \\ \vspace{0.5cm} \large{\thegruppe}}

\author{Martin Ueding \\ \small{\href{mailto:mu@uni-bonn.de}{mu@uni-bonn.de}} \and Simon Schlepphorst \\ \small{\href{mailto:s2@uni-bonn.de}{s2@uni-bonn.de}}}

\begin{document}

\maketitle

\begin{table}[h]
	\centering
	\begin{tabular}{l|c|c|c|c|c|c}
		Aufgabe & H \theuebung.1 & H \theuebung.2 & H \theuebung.3 & H \theuebung.4 & H \theuebung.5 & $\Sigma$   \\
		\hline
		Punkte & \punkte & \punkte & \punkte & \punkte & \punkte & \punkte
	\end{tabular}
\end{table}

%%%%%%%%%%%%%%%%%%%%%%%%%%%%%%%%%%%%%%%%%%%%%%%%%%%%%%%%%%%%%%%%%%%%%%%%%%%%%%%
%                  zylindrische und sphärische Koordinaten                   %
%%%%%%%%%%%%%%%%%%%%%%%%%%%%%%%%%%%%%%%%%%%%%%%%%%%%%%%%%%%%%%%%%%%%%%%%%%%%%%%

\section{zylindrische und sphärische Koordinaten}

\subsection{orthonormales System}

\paragraph{Kugelkoordinaten}

Die Transformationen von zylindrische in kartesische Koordinaten ist gegeben durch:
\[
	\vec\varphi\del{r, \phi, \theta} = r \begin{pmatrix}
		\cos(\phi) \sin(\theta) \\
		\sin(\phi) \sin(\theta) \\
		\cos(\theta)
	\end{pmatrix}
\]

Die dazugehörige Jakobimatrix ist:
\[
	\tens J_k := \Dif \vec\varphi = \begin{pmatrix}
		\cos(\phi) \sin(\theta) & -r \sin(\phi)\sin(\theta) & r \cos(\phi) \cos(\theta) \\
		\sin(\phi) \sin(\theta) & r \cos(\phi)\sin(\theta) & r \sin(\phi)\cos(\theta) \\
						 \cos(\theta) & 0 & - r \sin(\theta)
	\end{pmatrix}
\]

Die Einheitsvektoren sind sind die Spalten der Jakobimatrix, allerdings normiert. Diese sind:
\[
	\vec e_r = \begin{pmatrix}
		\cos(\phi) \sin(\theta) \\
		\sin(\phi) \sin(\theta) \\
		\cos(\theta)
	\end{pmatrix}
	, \quad
	\vec e_\phi = \begin{pmatrix}
		-\sin(\phi) \\
		\cos(\phi) \\
		0
	\end{pmatrix}
	, \quad
	\vec e_\theta = \begin{pmatrix}
		\cos(\phi) \cos(\theta) \\
		\sin(\phi) \cos(\theta) \\
		-\sin(\theta)
	\end{pmatrix}
\]

Diese sollen jetzt orthogonal sein. Dies bedeutet, dass gilt:
\[
	\lambda_1 \vec e_r + \lambda_2 \vec e_\phi + \lambda_3 \vec e_\theta = \vec 0
	\quad \Leftrightarrow \quad
	\forall i \in \set{1, 2, 3}\colon \lambda_i = 0
\]

Ich sehe, dass $\inner{\vec e_r}{\vec e_\phi} = 0$, $\inner{\vec e_\phi}{\vec e_\theta} = 0$, und $\inner{\vec e_\theta}{\vec e_r} = 0$. Somit sind sie paarweise ortogonal. Da es drei Stück gibt, bilden sie eine orthonormale Basis.

\paragraph{Zylinderkoordinaten}

Die Transformation ist:
\[
	\vec\varphi\del{\rho, \phi, z} = \begin{pmatrix}
		\rho \cos(\phi) \\
		\rho \sin(\phi) \\
		z
	\end{pmatrix}
\]

Die Jakobimatrix ist:
\[
	\tens J_z := \Dif \vec\varphi = \begin{pmatrix}
		\cos(\phi) & - \rho \sin(\phi) & 0 \\
		\sin(\phi) & \rho \cos(\phi) & 0 \\
						  0 & 0 & 1
	\end{pmatrix}
\]

Die (normierten) Einheitsvektoren sind dann:
\[
	\vec e_\rho = \begin{pmatrix}
		\cos(\phi) \\
		\sin(\phi) \\
		0
	\end{pmatrix}
	, \quad
	\vec e_\phi = \begin{pmatrix}
		-\sin(\phi) \\
		\cos(\phi) \\
		0
	\end{pmatrix}
	, \quad
	\vec e_z = \begin{pmatrix}
		0 \\
		0 \\
		1
	\end{pmatrix}
\]

Auch hier kann ich schnell sehen, dass die Vektoren paarweise orthogonal sind. Somit bilden sie auch eine orthonormale Basis.

\subsection{Ableitungsoperatoren}

Die partiellen Ableitungen kann ich mit der Jakobimatrix transformieren. Dabei gilt für die partiellen Ableitungen:
\[
	\begin{pmatrix}
		\partial_x \\ \partial_y \\ \partial_z
	\end{pmatrix} = \tens J_z \begin{pmatrix}
		\partial_\rho \\ \partial_\phi \\ \partial_z
	\end{pmatrix}
\]

\subsubsection{Zylinderkoordinaten}


Die partiellen Ableitungen in Zylinderkoordinaten sind also:
\[
	\begin{pmatrix}
		\partial_x \\ \partial_y \\ \partial_z
	\end{pmatrix} =
	\begin{pmatrix}
		\cos(\phi) \partial_\rho - \rho \sin(\phi) \partial_\phi \\
		\sin(\phi) \partial_\rho + \rho \cos(\phi) \partial_\phi \\
		\partial_z
	\end{pmatrix}
\]

Mit der inversen Jakobimatrix geht das auch in die andere Richtung:
\[
	\begin{pmatrix}
		\partial_\rho \\ \partial_\phi \\ \partial_z
	\end{pmatrix} =
	\begin{pmatrix}
		\frac{x}{\sqrt{x^2 + y^2}} \partial_x + \frac{y}{\sqrt{x^2 + y^2}} \partial_y \\
			\frac{-y}{\sqrt{x^2 + y^2}} \partial_x + \frac{x}{\sqrt{x^2 + y^2}} \partial_y \\
		\partial_z
	\end{pmatrix}
\]

\paragraph{Divergenz}

\fehlt

\paragraph{Rotation}

\fehlt

\paragraph{Laplace}

\fehlt

\subsubsection{Kugelkoordinaten}

Für die andere Richtung brauche ich zuerst die inverse Jakobimatrix:
\[
	\tens J_z^{-1} =
	\begin{pmatrix}
		\cos(\phi)\sin(\theta) & \cos(\theta) & \sin(\phi)\sin(\theta) \\
		\frac 1r \cos(\phi) \cos(\theta) & \frac 1r \sin(\phi) \cos(\theta) & - \frac 1r \sin(\theta) \\
		- \frac 1r \frac{\sin(\phi)}{\sin(\theta)} & \frac 1r \frac{\cos(\phi)}{\sin(\theta)} & 0
	\end{pmatrix}
\]

\paragraph{Divergenz}

\fehlt

\paragraph{Rotation}

\fehlt

\paragraph{Laplace}

\fehlt

%%%%%%%%%%%%%%%%%%%%%%%%%%%%%%%%%%%%%%%%%%%%%%%%%%%%%%%%%%%%%%%%%%%%%%%%%%%%%%%
%                         zum Integralsatz von Stokes                         %
%%%%%%%%%%%%%%%%%%%%%%%%%%%%%%%%%%%%%%%%%%%%%%%%%%%%%%%%%%%%%%%%%%%%%%%%%%%%%%%

\section{zum Integralsatz von Stokes}

Gegeben ist ein Vektorfeld:
\[ \vec K \del{\vec r} = \vec r \]

\subsection{Linienintegrale}

\paragraph{gerade Pfade}

Es soll ein Linienintegral berechnet werden, der Pfad ist in Abbildung \ref{fig:linien} dargestellt.

\begin{figure}[h]
	\centering
	\begin{minipage}[b]{0.4\textwidth}
		\centering
		\begin{tikzpicture}[scale=2.5]
			\draw[->>, thick] (0, 0) node[below left] {$(0, 0, 0)$} -- (0, 1) node[above left] {$(0, 0, 1)$};
			\draw[->>, thick] (0, 1) -- (1, 1) node[above right] {$(0, 1, 1)$};
			\draw[->>, thick] (1, 1) -- (0, 0);

			\foreach \x in {-0.2, 0, 0.2, 0.4, 0.8}
			\foreach \y in {-0.1, 0, 0.2, 0.5, 0.8}
			\draw[dotted, ->] (\x, \y) -- ++(\x, \y);
		\end{tikzpicture}
		\caption{Vektorfeld $\vec K$ und Pfad $\gamma_1$}
		\label{fig:linien}
	\end{minipage}
	\hspace{0.1\textwidth}
	\begin{minipage}[b]{0.4\textwidth}
		\centering
		\begin{tikzpicture}[scale=2.5]
			\draw[->>, thick] (0, 0) node[below left] {$(0, 0, 0)$} parabola (1, 1) node[above right] {$(0, 1, 1)$};
			\draw[->>, thick] (1, 1) -- (0, 0);

			\foreach \x in {-0.2, 0, 0.2, 0.4, 0.8}
			\foreach \y in {-0.1, 0, 0.2, 0.5, 0.8}
			\draw[dotted, ->] (\x, \y) -- ++(\x, \y);
		\end{tikzpicture}
		\caption{Vektorfeld $\vec K$ und Pfad $\gamma_2$}
		\label{fig:parabel}

	\end{minipage}

\end{figure}

Ich beginne mit den normierten Tangentialvektoren:
\[
	\vec \tau = \begin{pmatrix}
		0 \\ 0 \\ 1
	\end{pmatrix}
	,\quad
	\vec \tau = \begin{pmatrix}
		0 \\ 1 \\ 0
	\end{pmatrix}
	,\quad
	\vec \tau = \frac{1}{\sqrt{2}} \begin{pmatrix}
		0 \\ -1 \\ -1
	\end{pmatrix}
\]

Die Karten für die eindimensionale Untermannigfaltigkeit, die die Linie ja ist, sind:
\[
	\vec\varphi_1(t) = \begin{pmatrix}
		0 \\ 0 \\ 1
	\end{pmatrix} t
	,\quad
	\vec\varphi_1(t) = \begin{pmatrix}
		0 \\ 0 \\ 1
	\end{pmatrix} + \begin{pmatrix}
		0 \\ 1 \\ 0
	\end{pmatrix} t
	,\quad
	\vec\varphi_1(t) = \begin{pmatrix}
		0 \\ 1 \\ 1
	\end{pmatrix} + \frac{1}{\sqrt{2}} \begin{pmatrix}
		0 \\ -1 \\ -1
	\end{pmatrix} t
\]

Die gramschen Determinanten sind allesamt $1$.

Nun kann ich die Linienintegrale ausrechnen:
\begin{align*}
	W &=
	\sum_i \int \dif t \inner{\vec \tau_i}{\vec K \del{\vec\varphi_i(t)}} \\
	\intertext{Ich setze alles ein.}
	&=
	\int_0^1 \dif t \inner{\begin{pmatrix}
			0 \\ 0 \\ 1
	\end{pmatrix}}{\begin{pmatrix}
			0 \\ 0 \\ 1
	\end{pmatrix} t}
	+
	\int_0^1 \dif t \inner{\begin{pmatrix}
			0 \\ 1 \\ 0
	\end{pmatrix}}{\begin{pmatrix}
			0 \\ 0 \\ 1
		\end{pmatrix} + \begin{pmatrix}
			0 \\ 1 \\ 0
	\end{pmatrix} t} \\
	&\quad+
	\int_0^{\sqrt{2}} \dif t \inner{\frac 1{\sqrt{2}} \begin{pmatrix}
		0 \\ -1 \\ -1 \end{pmatrix}}{\begin{pmatrix}
				0 \\ 1 \\ 1
			\end{pmatrix} + \frac{1}{\sqrt{2}} \begin{pmatrix}
				0 \\ -1 \\ -1
		\end{pmatrix} t} \\
		\intertext{Die Skalarprodukte löse ich auf.}
		&= \int_0^1 \dif t t
		+\int_0^1 \dif t t
		+\int_0^{\sqrt{2}} \dif t \sqrt{2} \del{1-\frac t{\sqrt{2}}} \\
		&= \half + \half -1 = 0
	\end{align*}

	\paragraph{Parabel}

	Nun der zweite Pfad $\gamma_2$, diesmal mit Parabelbogen (Abbildung \ref{fig:parabel}).

	Für den Parabelteil ist meine Karte $\vec\varphi$, deren gramsche
	Determinante $g$ und der Tangentialvektor:
	\[
		\vec\varphi(t) = \begin{pmatrix}
			0 \\ t \\ t^2
		\end{pmatrix}
		, \quad
		g = 1 + 4t^2
		, \quad
		\vec \tau = \frac 1{\sqrt{g}} \begin{pmatrix}
			0 \\ 1 \\ 2t
		\end{pmatrix}
	\]

	Das Linienintegral ist nun, zusammen mit dem zweiten Pfadteil, dessen Ergebnis $-1$ ich aus der vorherigen Aufgabe übernehme:
	\begin{align*}
		W
		&= \int_0^1 \dif t \inner{\begin{pmatrix}
		0 \\ t \\ t^2
\end{pmatrix}}{\frac 1{\sqrt{1+4t^2}} \begin{pmatrix}
		0 \\ 1 \\ 2t
\end{pmatrix}} \sqrt{1+4t^2} - 1 \\
&= \sbr{\half t^2 + \half t^4}_0^1 - 1 = 0
\end{align*}

Es fällt auf, dass die beiden geschlossenen Integrale gerade null sind. Somit scheint dieses Kraftfeld, zumindest hier lokal, konservativ zu sein.

\subsection{mit dem Satz von Stokes}

Die geschlossenen Linienintegrale können auch über die Rotation des
Vektorfeldes bestimmt werden. Das Vektorfeld muss allerdings differenzierbar
sein und das Gebiet kompakt.

Da $\vnabla \times \vec r = 0$ gilt, sind hier alle geschlossenen Pfadintegrale
gleich null.

%%%%%%%%%%%%%%%%%%%%%%%%%%%%%%%%%%%%%%%%%%%%%%%%%%%%%%%%%%%%%%%%%%%%%%%%%%%%%%%
%                             Multipolentwicklung                             %
%%%%%%%%%%%%%%%%%%%%%%%%%%%%%%%%%%%%%%%%%%%%%%%%%%%%%%%%%%%%%%%%%%%%%%%%%%%%%%%

\section{Multipolentwicklung}

Die Verteilungen der Ladungen im Raum habe ich in Abbildung \ref{fig:ladungen}
skizziert. Berechnet werden soll das Dipolmoment $\vec p$ und der Quadrupoltensor $\tens Q$ bestimmt werden.

\begin{figure}[h]
	\centering
	\begin{tikzpicture}[scale=3]
		\foreach \x in {-1, 0, 1}
		\foreach \y in {-1, 0, 1}
		\foreach \z in {-1, 0, 1}
		\draw[dotted] (\x, \y, \z) -- ++(1, 0, 0);

		\foreach \x in {-1, 0, 1, 2}
		\foreach \y in {-1, 0}
		\foreach \z in {-1, 0, 1}
		\draw[dotted] (\x, \y, \z) -- ++(0, 1, 0);

		\foreach \x in {-1, 0, 1, 2}
		\foreach \y in {-1, 0, 1}
		\foreach \z in {-1, 0}
		\draw[dotted] (\x, \y, \z) -- ++(0, 0, 1);

		\draw (0, 1, 0) -- (0, 0, 1) -- (0, -1, 0) -- (0, 0, -1) -- cycle;
		\draw (2, 0, 0) -- (-1, 0, 0);

		\shade[ball color=black!5] (   0,  1,  0) circle (1.2mm) node {$+q$};
		\shade[ball color=black!5] (   0, -1,  0) circle (1.2mm) node {$+q$};
		\shade[ball color=black!5] (   0,  0,  1) circle (1.2mm) node {$+q$};
		\shade[ball color=black!5] (   0,  0, -1) circle (1.2mm) node {$+q$};
		\shade[ball color=black!5] (  -1,  0,  0) circle (1.2mm) node {$-q$};
		\shade[ball color=black!5] (-0.5,  0,  0) circle (1.2mm) node {$-q$};
		\shade[ball color=black!5] (   1,  0,  0) circle (1.2mm) node {$-q$};
		\shade[ball color=black!5] (   2,  0,  0) circle (1.2mm) node {$-q$};
	\end{tikzpicture}
	\caption{Anordnung der Ladungen}
	\label{fig:ladungen}
\end{figure}

Das Dipolmoment $\vec p$ eines Dipols mit Ladung $+q$ und $-q$ und Separation
$\vec d$ ist:
\[
	\vec p = q \vec d
\]

\fehlt

%%%%%%%%%%%%%%%%%%%%%%%%%%%%%%%%%%%%%%%%%%%%%%%%%%%%%%%%%%%%%%%%%%%%%%%%%%%%%%%
%                Magnetisches Feld eines ausgedehnten Leiters                 %
%%%%%%%%%%%%%%%%%%%%%%%%%%%%%%%%%%%%%%%%%%%%%%%%%%%%%%%%%%%%%%%%%%%%%%%%%%%%%%%

\section{Magnetisches Feld eines ausgedehnten Leiters}

Es gilt $\vnabla \times \vec H = \vec j$. Dies kann ich auch in Integralen
schreiben, für dieses Problem benutze ich Zylinderkoordinaten. Ich betrachte
eine Kreisfläche $A$ mit Radius $r$, die den Leiter in einem Kreis schneidet.
Das Normalenvektorfeld auf $A$ ist $\vec \nu$, der Tangentialvektor an den
Kreisrand ist $\vec \tau$. Die Stromdichte $\vec j$ ist $\frac{\vec I}{\pi R^2}$ Somit gilt nach dem Satz von Stokes:
%
\[
	\oint_{\partial A} \dif l \inner{\vec H}{\vec \tau} = \int_A \dif A \inner{\vec j}{\vec \nu}
\]

Die Integrale sind für den Fall $r > R$ durch Wahl des Koordinatensystems trivial. Somit folgt für das magnetische Feld:
\[
	2 \pi r H = I
	\quad \Leftrightarrow \quad
	H(r) = \frac{1}{2\pi} \frac{1}r I
\]

Falls $r < R$ gilt, ist der umschlossene Strom kleiner. Dann gilt nur noch:
\[
	2 \pi r H = \frac{r^2}{R^2} I
	\quad \Leftrightarrow \quad
	H(r) = \frac{1}{2\pi} \frac{r}{R^2} I
\]

Das Feld ist ein Wirbelfeld, dessen Richtung mit der rechten Handregel bestimmt
werden kann.

%%%%%%%%%%%%%%%%%%%%%%%%%%%%%%%%%%%%%%%%%%%%%%%%%%%%%%%%%%%%%%%%%%%%%%%%%%%%%%%
%                       geladene Walze, Rohr und Platte                       %
%%%%%%%%%%%%%%%%%%%%%%%%%%%%%%%%%%%%%%%%%%%%%%%%%%%%%%%%%%%%%%%%%%%%%%%%%%%%%%%

\section{geladene Walze, Rohr und Platte}

Bei all diesen Aufgaben benutze ich das Gesetz von Gauß.

\subsection{Hohlzylinder}

Die Oberflächenladungsdichte auf der Außenseite sei $\sigma$. Als
Integrationsvolumen benutze ich einen Zylinder mit Radius $r$, der konzentrisch mit dem
Hohlzylinder liegt. Dann gilt:
\[
	\oint_{\partial V} \dif A \inner{\vec D}{\vec \nu} = \int_V \dif V \rho
\]

Die Integrale sind hier recht trivial:
\[
	2 \pi r h D = 2 \pi R h \sigma
	\quad \Leftrightarrow \quad
	\vec D = \sigma \frac{R}{r} \hat{\vec r}
\]

Innerhalb des Zylinders befindet sich keine Ladung, so dass es dort auch kein Feld bilden kann.

Das Potential ist das negative Feld nach dem Radius integriert. Dabei ist es innerhalb des Zylinders konstant, weil keine Kraft wirkt.
\[
	\phi(r) = \begin{cases}
		\epsilon_0 R \sigma \ln\del{\frac rR} & r \geq R \\
								 \phi(R) & r < R
	\end{cases}
\]

Verschiebungsdichte und Potential sind in Abbildung \ref{fig:Dhohl} beziehungsweise \ref{fig:phihohl} gezeigt.

\begin{figure}[h]
	\centering
	\begin{minipage}[b]{0.45\textwidth}
		\centering
		\begin{tikzpicture}
			\draw[domain=1:3] plot (\x, {1 / \x});
			\draw[thin, ->] (0, -1) -- ++(3, 0) node[right] {$r$};
			\draw[thin, ->] (0, -1) -- ++(0, +1.5) node[above] {$D(r)$};
			\draw[thin] (1, -0.9) -- ++(0, -0.2) node[below] {$R$};
		\end{tikzpicture}
		\caption{Verschiebungsdichte des Hohlzylinders}
		\label{fig:Dhohl}
	\end{minipage}
	\begin{minipage}[b]{0.45\textwidth}
		\centering
		\begin{tikzpicture}
			\draw[domain=1:3] plot (\x, {ln(\x)});
			\draw (0, 0) -- (1, 0);
			\draw[thin, ->] (0, -1) -- ++(3, 0) node[right] {$r$};
			\draw[thin, ->] (0, -1) -- ++(0, +1.5) node[above] {$\phi(r)$};
			\draw[thin] (1, -0.9) -- ++(0, -0.2) node[below] {$R$};
		\end{tikzpicture}
		\caption{Potential des Hohlzylinders}
		\label{fig:phihohl}
	\end{minipage}
\end{figure}

\subsection{Vollzylinder}

Der Vollzylinder funktioniert analog für den Fall $r > R$. Nur wird die Fläche und Oberflächenladungsdichte durch Volumen und Ladungsdichte ersetzt.
\[
	2 \pi r h D = \pi R^2 h \rho
	\quad \Leftrightarrow \quad
	\vec D = \half \rho \frac{R^2}{r} \hat{\vec r}
\]

Innerhalb des Vollzylinders gibt es nun ein Feld. Dieses ist:
\[
	2 \pi r h D = \pi r^2 h \rho
	\quad \Leftrightarrow \quad
	\vec D = \half \rho r \hat{\vec r}
\]

Das Potential ist wieder das negative Feld nach dem Radius integriert. Es ergibt sich:
\[
	\phi(r) = \begin{cases}
		\half \rho R^2 \ln\del{\frac rR} & r \geq R \\
   \frac 14 \rho r^2 - \frac 14 \rho R^2 & r < R
	\end{cases}
\]

Verschiebungsdichte und Potential sind in Abbildung \ref{fig:Dvoll} beziehungsweise \ref{fig:phivoll} gezeigt.

\begin{figure}[h]
	\centering
	\begin{minipage}[b]{0.45\textwidth}
		\centering
		\begin{tikzpicture}
			\draw[domain=1:3] plot (\x, {0.5 / \x});
			\draw[domain=0:1] plot (\x, {0.5 * \x});
			\draw[thin, ->] (0, -1) -- ++(3, 0) node[right] {$r$};
			\draw[thin, ->] (0, -1) -- ++(0, +1.5) node[above] {$D(r)$};
			\draw[thin] (1, -0.9) -- ++(0, -0.2) node[below] {$R$};
		\end{tikzpicture}
		\caption{Verschiebungsdichte des Vollzylinders}
		\label{fig:Dvoll}
	\end{minipage}
	\begin{minipage}[b]{0.45\textwidth}
		\centering
		\begin{tikzpicture}
			\draw[domain=1:3] plot (\x, {0.5 * ln(\x)});
			\draw[domain=0:1] plot (\x, {0.25 * \x^2 - 0.25});
			\draw[thin, ->] (0, -1) -- ++(3, 0) node[right] {$r$};
			\draw[thin, ->] (0, -1) -- ++(0, +1.5) node[above] {$\phi(r)$};
			\draw[thin] (1, -0.9) -- ++(0, -0.2) node[below] {$R$};
		\end{tikzpicture}
		\caption{Potential des Vollzylinders}
		\label{fig:phivoll}
	\end{minipage}
\end{figure}

\subsection{Platte}

Bei einer Platte wähle ich ein beliebiges Prisma, dessen Stirnflächen (Flächeninhalt $A$) parallel zur Platte sind und das von der Platte geschnitten wird. Aus der Symmetrie folgt dann, dass kein Fluss durch die Seitenflächen geht. Nach dem Gesetz von Gauß gilt dann:
\[
	2 A D = A \sigma
	\quad \Leftrightarrow \quad
	\vec D = \half \sigma \hat{\vec r}
\]

Das Potential ist das negative Feld nach dem Abstand integriert:
\[
	\phi(r) = - \half \sigma \abs r
\]

Verschiebungsdichte und Potential sind in Abbildung \ref{fig:Dplatte} beziehungsweise \ref{fig:phiplatte} gezeigt.

\begin{figure}[h]
	\centering
	\begin{minipage}[b]{0.45\textwidth}
		\centering
		\begin{tikzpicture}
			\draw (-1, -1) -- (0, -1);
			\draw (1, 1) -- (0, 1);
			\draw[thin, ->] (-1, 0) -- (1, 0) node[right] {$r$};
			\draw[thin, ->] (0, -1.3) -- (0, +1.3) node[above] {$D(r)$};
		\end{tikzpicture}
		\caption{Verschiebungsdichte der Platte}
		\label{fig:Dplatte}
	\end{minipage}
	\begin{minipage}[b]{0.45\textwidth}
		\centering
		\begin{tikzpicture}
			\draw[domain=-1:1] plot (\x, {-abs(\x)});
			\draw[thin, ->] (-1, 0) -- (1, 0) node[right] {$r$};
			\draw[thin, ->] (0, 0) -- ++(0, +1) node[above] {$\phi(r)$};
		\end{tikzpicture}
		\caption{Potential der Platte}
		\label{fig:phiplatte}
	\end{minipage}
\end{figure}

\end{document}

% vim: spell spelllang=de
