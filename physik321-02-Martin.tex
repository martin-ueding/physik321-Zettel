% Copyright © 2012 Martin Ueding <dev@martin-ueding.de>
%
\documentclass[11pt, ngerman, fleqn]{article}

\usepackage[a4paper, left=3cm, right=2cm, top=2cm, bottom=2cm]{geometry}
\usepackage[activate]{pdfcprot}
\usepackage[cdot, squaren]{SIunits}
\usepackage[iso]{isodate}
\usepackage[parfill]{parskip}
\usepackage[T1]{fontenc}
\usepackage[utf8]{inputenc}
\usepackage{amsmath}
\usepackage{amssymb}
\usepackage{amsthm}
\usepackage{babel}
\usepackage{color}
\usepackage{commath}
\usepackage{epstopdf}
\usepackage{fancyhdr}
\usepackage{graphicx}
\usepackage{hyperref}
\usepackage{setspace}
\usepackage{tikz}

\usepackage[charter]{mathdesign}

\definecolor{darkblue}{rgb}{0,0,.5}
\definecolor{darkgreen}{rgb}{0,.5,0}

\hypersetup{
	breaklinks=false,
	citecolor=darkgreen,
	colorlinks=true,
	linkcolor=black,
	menucolor=black,
	urlcolor=darkblue,
}

\setlength{\columnsep}{2cm}

\DeclareMathOperator{\arcsinh}{arsinh}
\DeclareMathOperator{\arsinh}{arsinh}
\DeclareMathOperator{\asinh}{arsinh}
\DeclareMathOperator{\card}{card}
\DeclareMathOperator{\diam}{diam}

\newcommand{\dalambert}{\mathop{{}\Box}\nolimits}
\newcommand{\divergence}[1]{\inner{\vnabla}{#1}}
\newcommand{\ee}{\mathrm e}
\newcommand{\emesswert}{\del{\messwert \pm \messwert}}
\newcommand{\ev}{\hat{\vec e}}
\newcommand{\e}[1]{\cdot 10^{#1}}
\newcommand{\fehlt}{\textcolor{red}{Hier fehlen noch Inhalte.}}
\newcommand{\half}{\frac 12}
\newcommand{\ii}{\mathrm i}
\newcommand{\inner}[2]{\left\langle #1, #2 \right\rangle}
\newcommand{\laplace}{\mathop{{}\bigtriangleup}\nolimits}
\newcommand{\messwert}{\textcolor{blue}{\square}}
\newcommand{\punkte}{\textcolor{white}{xxxxx}}
\newcommand{\tens}[1]{\boldsymbol{#1}}
\newcommand{\vnabla}{\vec \nabla}
\renewcommand{\vec}[1]{\boldsymbol{#1}}

\newcommand{\themodul}{physik321}
\newcommand{\thegruppe}{Gruppe 8 -- Julia Volmer}
\newcommand{\theuebung}{2}

\pagestyle{fancy}

\fancyfoot[C]{\footnotesize{\thegruppe}}
\fancyfoot[L]{\footnotesize{Martin Ueding, Simon Schlepphorst}}
\fancyfoot[R]{\footnotesize{Seite \thepage}}
\fancyhead[L]{\themodul{} -- Übung \theuebung}

\setcounter{section}{0}

\def\thesection{H \theuebung.\arabic{section}}
\def\thesubsection{\thesection\alph{subsection}}

\title{\themodul{} -- Übung \theuebung \\ \vspace{0.5cm} \large{\thegruppe}}

\author{Martin Ueding \\ \small{\href{mailto:mu@uni-bonn.de}{mu@uni-bonn.de}} \and Simon Schlepphorst \\ \small{\href{mailto:s2@uni-bonn.de}{s2@uni-bonn.de}}}

\begin{document}

\maketitle

\begin{table}[h]
	\centering
	\begin{tabular}{l|c|c|c|c|c|c}
		Aufgabe & \ref{1} & \ref{2} & \ref{3} & \ref{4} & \ref{5} & $\Sigma$   \\
		\hline
		Punkte & \punkte / 10 & \punkte / 10 & \punkte / 10 & \punkte / 10 & \punkte / 10 & \punkte / 40
	\end{tabular}
\end{table}

%%%%%%%%%%%%%%%%%%%%%%%%%%%%%%%%%%%%%%%%%%%%%%%%%%%%%%%%%%%%%%%%%%%%%%%%%%%%%%%
%                  zylindrische und sphärische Koordinaten                   %
%%%%%%%%%%%%%%%%%%%%%%%%%%%%%%%%%%%%%%%%%%%%%%%%%%%%%%%%%%%%%%%%%%%%%%%%%%%%%%%

\section{zylindrische und sphärische Koordinaten}
\label{1}

\subsection{orthonormales System}

\paragraph{Kugelkoordinaten}

Die Transformationen von zylindrische in kartesische Koordinaten ist gegeben durch:
\[
	\vec\varphi\del{r, \phi, \theta} = r \begin{pmatrix}
		\cos(\phi) \sin(\theta) \\
		\sin(\phi) \sin(\theta) \\
		\cos(\theta)
	\end{pmatrix}
\]

Die dazugehörige Jakobimatrix ist:
\[
	\tens J_k := \Dif \vec\varphi = \begin{pmatrix}
		\cos(\phi) \sin(\theta) & -r \sin(\phi)\sin(\theta) & r \cos(\phi) \cos(\theta) \\
		\sin(\phi) \sin(\theta) & r \cos(\phi)\sin(\theta) & r \sin(\phi)\cos(\theta) \\
						 \cos(\theta) & 0 & - r \sin(\theta)
	\end{pmatrix}
\]

Die Einheitsvektoren sind sind die Spalten der Jakobimatrix, allerdings normiert. Diese sind:
\[
	\vec e_r = \begin{pmatrix}
		\cos(\phi) \sin(\theta) \\
		\sin(\phi) \sin(\theta) \\
		\cos(\theta)
	\end{pmatrix}
	, \quad
	\vec e_\phi = \begin{pmatrix}
		-\sin(\phi) \\
		\cos(\phi) \\
		0
	\end{pmatrix}
	, \quad
	\vec e_\theta = \begin{pmatrix}
		\cos(\phi) \cos(\theta) \\
		\sin(\phi) \cos(\theta) \\
		-\sin(\theta)
	\end{pmatrix}
\]

Diese sollen jetzt orthogonal sein. Dies bedeutet, dass gilt:
\[
	\lambda_1 \vec e_r + \lambda_2 \vec e_\phi + \lambda_3 \vec e_\theta = \vec 0
	\quad \Leftrightarrow \quad
	\forall i \in \set{1, 2, 3}\colon \lambda_i = 0
\]

Ich sehe, dass $\inner{\vec e_r}{\vec e_\phi} = 0$, $\inner{\vec e_\phi}{\vec e_\theta} = 0$, und $\inner{\vec e_\theta}{\vec e_r} = 0$. Somit sind sie paarweise ortogonal. Da es drei Stück gibt, bilden sie eine orthonormale Basis.

\paragraph{Zylinderkoordinaten}

Die Transformation ist:
\[
	\vec\varphi\del{\rho, \phi, z} = \begin{pmatrix}
		\rho \cos(\phi) \\
		\rho \sin(\phi) \\
		z
	\end{pmatrix}
\]

Die Jakobimatrix ist:
\[
	\tens J_z := \Dif \vec\varphi = \begin{pmatrix}
		\cos(\phi) & - \rho \sin(\phi) & 0 \\
		\sin(\phi) & \rho \cos(\phi) & 0 \\
						  0 & 0 & 1
	\end{pmatrix}
\]

Die (normierten) Einheitsvektoren sind dann:
\[
	\vec e_\rho = \begin{pmatrix}
		\cos(\phi) \\
		\sin(\phi) \\
		0
	\end{pmatrix}
	, \quad
	\vec e_\phi = \begin{pmatrix}
		-\sin(\phi) \\
		\cos(\phi) \\
		0
	\end{pmatrix}
	, \quad
	\vec e_z = \begin{pmatrix}
		0 \\
		0 \\
		1
	\end{pmatrix}
\]

Auch hier kann ich schnell sehen, dass die Vektoren paarweise orthogonal sind. Somit bilden sie auch eine orthonormale Basis.

\subsection{Ableitungsoperatoren}

Die Vorgehensweise habe ich in \cite{lautrup-curviliniar} gefunden.

\subsubsection{Zylinderkoordinaten}

Für den Nabla in Zylinderkoordinaten projiziere ich diesen auf die
Einheitsvektoren, die allerdings nicht normiert worden sind. Von der anderen
Seite kann ich über die Kettenregel die Koordinatentransformation durchführen.
\begin{align*}
	\dpd{}\rho &= \dpd x\rho \dpd{}x + \dpd y\rho \dpd{}y = \cos(\phi) \dpd{}x + \sin(\phi) \dpd{}y = \inner{\ev_\rho}\vnabla \\
 \dpd{}\phi &= \dpd x\phi \dpd{}x + \dpd y\phi \dpd{}y= - \rho \sin(\phi) \dpd{}x + \rho \cos(\phi) \dpd{}y = \rho \inner{\ev_\phi}\vnabla \\
	  \dpd{} z &= \dpd{} z = \inner{\hat{\vec z}_\phi}\vnabla
\end{align*}

Daraus folgt:
\[
	\vnabla_z :=
	\ev_\rho
		\dpd{}\rho
	+\ev_\phi
		\frac 1\rho \dpd{}\phi
	+\ev_z
		\dpd{}z
\]

\paragraph{Divergenz}

Nun wende ich den gerade hergeleiteten $\vnabla_z$-Operator auf ein Vektorfeld $\vec A$ an, das in Zylinderkoordinaten gegeben ist:
\[
	\vec A := \ev_\rho A_\rho + \ev_\phi A_\phi + \ev_z A_z
\]

Anders als in den kartesischen Koordinaten sind die Ableitungen der Basisvektoren nicht immer 0:
\[
	\dpd{\ev_\rho}\phi = \ev_\phi
	, \quad
	\dpd{\ev_\phi}\phi = - \ev_\rho
\]

Bei der Divergenz ist zu beachten, dass nicht nur das Vektorfeld, sondern auch
die Einheitsvektoren abgeleitet werden müssen. Das Skalarprodukt
$\inner{\vnabla_z}{\vec A}$ wird dann zu einem normalen Produkt. Im
kartesischen Fall würden dann einfach alle Einheitsvektoren, die nicht gleich
sind, als Skalarprodukt 0 haben und entfallen. Mit den beiden Identitäten oben, kommt allerdings noch ein weiterer Term hinzu:
\begin{align*}
	\inner{\vnabla_z}{\vec A}
	&=
	\ev_\rho \dpd{}\rho \del{\ev_\rho A_\rho + \ev_\phi A_\phi + \ev_z A_z} \\
	&\quad+\ev_\phi \frac 1\rho \dpd{}\phi \del{\ev_\rho A_\rho + \ev_\phi A_\phi + \ev_z A_z} \\
	&\quad+\ev_z \dpd{}z \del{\ev_\rho A_\rho + \ev_\phi A_\phi + \ev_z A_z} \\
	\intertext{Die meisten Skalarprodukte entfallen direkt. Bei den $\pd{}\phi$ Termen wandeln sich einige Einheitsvektoren um, so dass deren Skalarprodukt mit $\ev_\phi$ nicht mehr 0 sind.}
	&=
	\dpd{A_\rho}\rho + \ev_\phi \frac 1\rho \del{
		\ev_\phi A_\rho - \ev_\rho A_\phi + \ev_\phi \dpd{A_\phi}\phi
	} + \dpd{A_z}z \\
	&=
	\dpd{A_\rho}\rho + \frac{A_\rho}\rho  + \frac 1\rho \dpd{A_\phi}\phi
	 + \dpd{A_z}z
\end{align*}

\paragraph{Rotation}

Die Rotation funktioniert ähnlich wie die Divergenz.
\begin{align*}
	\vnabla_z \times A
	&=
	\begin{vmatrix}
		\pd{}\rho & \frac 1\rho \pd{}\phi & \pd{}z \\
		   A_\rho & A_\phi & A_z \\
		 \ev_\rho & \ev_\phi & \ev_z
	\end{vmatrix} \\
	&=
	\dpd{}\rho \del{A_\phi \ev_z - A_z \ev_\phi}
	+ \frac 1\rho \dpd{}\phi \del{A_z \ev_\rho - A_\rho \ev_z}
	+ \dpd{}z \del{A_\rho \ev_\phi - A_\phi \ev_\rho} \\
	\intertext{Auch hier kommen durch die Produktregel einige weitere Terme hinzu.}
	&=
	\ev_z \dpd{A_\phi}\rho - \ev_\phi \dpd{A_z}\rho
	+ \frac 1\rho \del{\ev_\rho \dpd{A_z}\phi + \ev_\phi A_z - \ev_z \dpd{A_\rho}\phi}
	+ \ev_\phi \dpd{A_\rho}z - \ev_\rho \dpd{A_\phi}z \\
	&=
	\ev_\rho \del{-\dpd{A_\phi}z
		+ \frac 1\rho  \dpd{A_z}\phi
	}
	+ \ev_\phi \del{ \dpd{A_\rho}z
		- \dpd{A_z}\rho
		+ \frac 1\rho A_z
	}
	+ \ev_z \del{- \frac 1\rho \dpd{A_\rho}\phi
		+ \dpd{A_\phi}\rho
	}
\end{align*}

Das Ergebnis ist nicht ganz das Richtige, allerdings kann ich den Fehler nicht
finden.

\paragraph{Laplace}

In die Gleichung mit der Divergenz kann ich den Gradienten einsetzen. Ich erhalte dann:
\[
	\laplace_z = \frac 1\rho \dpd{}\rho \del{\rho \dpd {}\rho} + \frac 1{\rho^2} \dpd[2]{}\phi + \dpd[2]{}z
\]

\subsubsection{Kugelkoordinaten}

Die gleiche Idee wie bei den Zylinderkoordinaten:
\begin{align*}
	\dpd{}r &= \dpd xr \dpd{}x + \dpd yr \dpd{}y + \dpd zr \dpd{}z = \cos(\phi) \sin(\theta) \dpd{}x + \sin(\phi) \sin(\theta) \dpd{}y + \cos(\theta) \dpd{}z = \inner{\ev_r}\vnabla \\
	%
 \dpd{}\phi &= \dpd x\phi \dpd{}x + \dpd y\phi \dpd{}y + \dpd z\phi \dpd{}z = - r \sin(\phi)\sin(\theta) \dpd{}x + r \cos(\phi) \sin(\theta) \dpd{}y = \frac 1{r \sin(\theta)} \inner{\ev_\phi}\vnabla \\
	%
  \dpd{}\theta &= \dpd x\theta \dpd{}x + \dpd y\theta \dpd{}y + \dpd z\theta \dpd{}z = r \cos(\phi) \cos(\theta) \dpd{}x + r \sin(\phi) \cos(\theta) \dpd{}y - r \sin(\theta) \dpd{}z \\
			   &= \frac 1r \inner{\ev_\theta}\vnabla
\end{align*}

Daraus folgt:
\[
	\vnabla_k :=
	\ev_r
		\dpd{}r
	+\ev_\phi
		\frac 1{r \sin(\theta)} \dpd{}\phi
	+\ev_\theta
		\frac 1r \dpd{}\theta
\]

Wie bei den Zylinderkoordinaten wird hier eingesetzt und die Basisvektoren mit
differenziert. Diese Rechnung lasse ich aus.

%%%%%%%%%%%%%%%%%%%%%%%%%%%%%%%%%%%%%%%%%%%%%%%%%%%%%%%%%%%%%%%%%%%%%%%%%%%%%%%
%                         zum Integralsatz von Stokes                         %
%%%%%%%%%%%%%%%%%%%%%%%%%%%%%%%%%%%%%%%%%%%%%%%%%%%%%%%%%%%%%%%%%%%%%%%%%%%%%%%

\section{zum Integralsatz von Stokes}
\label{2}

Gegeben ist ein Vektorfeld:
\[ \vec K \del{\vec r} = \vec r \]

\subsection{Linienintegrale}

\paragraph{gerade Pfade}

Es soll ein Linienintegral berechnet werden, der Pfad ist in Abbildung \ref{fig:linien} dargestellt.

\begin{figure}[h]
	\centering
	\begin{minipage}[b]{0.4\textwidth}
		\centering
		\begin{tikzpicture}[scale=2.5]
			\draw[->>, thick] (0, 0) node[below left] {$(0, 0, 0)$} -- (0, 1) node[above left] {$(0, 0, 1)$};
			\draw[->>, thick] (0, 1) -- (1, 1) node[above right] {$(0, 1, 1)$};
			\draw[->>, thick] (1, 1) -- (0, 0);

			\foreach \x in {-0.2, 0, 0.2, 0.4, 0.8}
			\foreach \y in {-0.1, 0, 0.2, 0.5, 0.8}
			\draw[dotted, ->] (\x, \y) -- ++(\x, \y);
		\end{tikzpicture}
		\caption{Vektorfeld $\vec K$ und Pfad $\gamma_1$}
		\label{fig:linien}
	\end{minipage}
	\hspace{0.1\textwidth}
	\begin{minipage}[b]{0.4\textwidth}
		\centering
		\begin{tikzpicture}[scale=2.5]
			\draw[->>, thick] (0, 0) node[below left] {$(0, 0, 0)$} parabola (1, 1) node[above right] {$(0, 1, 1)$};
			\draw[->>, thick] (1, 1) -- (0, 0);

			\foreach \x in {-0.2, 0, 0.2, 0.4, 0.8}
			\foreach \y in {-0.1, 0, 0.2, 0.5, 0.8}
			\draw[dotted, ->] (\x, \y) -- ++(\x, \y);
		\end{tikzpicture}
		\caption{Vektorfeld $\vec K$ und Pfad $\gamma_2$}
		\label{fig:parabel}

	\end{minipage}

\end{figure}

Ich beginne mit den normierten Tangentialvektoren:
\[
	\vec \tau_1 = \begin{pmatrix}
		0 \\ 0 \\ 1
	\end{pmatrix}
	,\quad
	\vec \tau_2 = \begin{pmatrix}
		0 \\ 1 \\ 0
	\end{pmatrix}
	,\quad
	\vec \tau_3 = \frac{1}{\sqrt{2}} \begin{pmatrix}
		0 \\ -1 \\ -1
	\end{pmatrix}
\]

Die Karten für die eindimensionale Untermannigfaltigkeit, die die Linie ja ist, sind:
\[
	\vec\varphi_1(t) = \begin{pmatrix}
		0 \\ 0 \\ 1
	\end{pmatrix} t
	,\quad
	\vec\varphi_2(t) = \begin{pmatrix}
		0 \\ 0 \\ 1
	\end{pmatrix} + \begin{pmatrix}
		0 \\ 1 \\ 0
	\end{pmatrix} t
	,\quad
	\vec\varphi_3(t) = \begin{pmatrix}
		0 \\ 1 \\ 1
	\end{pmatrix} + \frac{1}{\sqrt{2}} \begin{pmatrix}
		0 \\ -1 \\ -1
	\end{pmatrix} t
\]

Die Definitionsbereiche der Karten sind $(0, 1)$, $(0, 1)$ und $\intoo{0, \sqrt{2}}$. Die Randpunkte, die eigentlich zum Pfad noch dazu gehören sind eine Nullmenge und machen bei der Integration keinen Unterschied. Die gramschen Determinanten der Karten sind allesamt $1$.

Nun kann ich die Linienintegrale ausrechnen:
\begin{align*}
	W &=
	\sum_{i=1}^3 \int \dif t \inner{\vec \tau_i}{\vec K \del{\vec\varphi_i(t)}} \\
	\intertext{Ich setze alles ein.}
	&=
	\int_0^1 \dif t \inner{\begin{pmatrix}
			0 \\ 0 \\ 1
	\end{pmatrix}}{\begin{pmatrix}
			0 \\ 0 \\ 1
	\end{pmatrix} t}
	+
	\int_0^1 \dif t \inner{\begin{pmatrix}
			0 \\ 1 \\ 0
	\end{pmatrix}}{\begin{pmatrix}
			0 \\ 0 \\ 1
		\end{pmatrix} + \begin{pmatrix}
			0 \\ 1 \\ 0
	\end{pmatrix} t} \\
	&\quad+
	\int_0^{\sqrt{2}} \dif t \inner{\frac 1{\sqrt{2}} \begin{pmatrix}
		0 \\ -1 \\ -1 \end{pmatrix}}{\begin{pmatrix}
				0 \\ 1 \\ 1
			\end{pmatrix} + \frac{1}{\sqrt{2}} \begin{pmatrix}
				0 \\ -1 \\ -1
		\end{pmatrix} t} \\
		\intertext{Die Skalarprodukte löse ich auf.}
		&= \int_0^1 \dif t t
		+\int_0^1 \dif t t
		+\int_0^{\sqrt{2}} \dif t \sqrt{2} \del{1-\frac t{\sqrt{2}}} \\
		&= \half + \half -1 = 0
	\end{align*}

	\paragraph{Parabel}

	Nun der zweite Pfad $\gamma_2$, diesmal mit Parabelbogen (Abbildung \ref{fig:parabel}).

	Für den Parabelteil ist meine Karte $\vec\varphi$, deren gramsche
	Determinante $g$ und der Tangentialvektor:
	\[
		\vec\varphi(t) = \begin{pmatrix}
			0 \\ t \\ t^2
		\end{pmatrix}
		, \quad
		g = 1 + 4t^2
		, \quad
		\vec \tau = \frac 1{\sqrt{g}} \begin{pmatrix}
			0 \\ 1 \\ 2t
		\end{pmatrix}
	\]

	Das Linienintegral ist nun, zusammen mit dem zweiten Pfadteil, dessen Ergebnis $-1$ ich aus der vorherigen Aufgabe übernehme:
	\begin{align*}
		W
		&= \int_0^1 \dif t \inner{\begin{pmatrix}
		0 \\ t \\ t^2
\end{pmatrix}}{\frac 1{\sqrt{1+4t^2}} \begin{pmatrix}
		0 \\ 1 \\ 2t
\end{pmatrix}} \sqrt{1+4t^2} - 1 \\
&= \sbr{\half t^2 + \half t^4}_0^1 - 1 = 0
\end{align*}

Es fällt auf, dass die beiden geschlossenen Integrale gerade 0 sind. Somit scheint dieses Kraftfeld, zumindest hier lokal, konservativ zu sein.

\subsection{mit dem Satz von Stokes}

Die geschlossenen Linienintegrale können auch über die Rotation des
Vektorfeldes bestimmt werden. Das Vektorfeld muss allerdings differenzierbar
sein und das Gebiet kompakt.

Da $\vnabla \times \vec r = 0$ gilt, sind hier alle geschlossenen Pfadintegrale
gleich 0.

%%%%%%%%%%%%%%%%%%%%%%%%%%%%%%%%%%%%%%%%%%%%%%%%%%%%%%%%%%%%%%%%%%%%%%%%%%%%%%%
%                             Multipolentwicklung                             %
%%%%%%%%%%%%%%%%%%%%%%%%%%%%%%%%%%%%%%%%%%%%%%%%%%%%%%%%%%%%%%%%%%%%%%%%%%%%%%%

\section{Multipolentwicklung}
\label{3}

Die Verteilungen der Ladungen im Raum habe ich in Abbildung \ref{fig:ladungen}
skizziert. Berechnet werden soll das Dipolmoment $\vec p$ und der Quadrupoltensor $\tens Q$ bestimmt werden.

\begin{figure}[h]
	\centering
	\begin{tikzpicture}[scale=3]
		\foreach \x in {-1, 0, 1}
		\foreach \y in {-1, 0, 1}
		\foreach \z in {-1, 0, 1}
		\draw[dotted] (\x, \y, \z) -- ++(1, 0, 0);

		\foreach \x in {-1, 0, 1, 2}
		\foreach \y in {-1, 0}
		\foreach \z in {-1, 0, 1}
		\draw[dotted] (\x, \y, \z) -- ++(0, 1, 0);

		\foreach \x in {-1, 0, 1, 2}
		\foreach \y in {-1, 0, 1}
		\foreach \z in {-1, 0}
		\draw[dotted] (\x, \y, \z) -- ++(0, 0, 1);

		\draw (0, 1, 0) -- (0, 0, 1) -- (0, -1, 0) -- (0, 0, -1) -- cycle;
		\draw (2, 0, 0) -- (-1, 0, 0);

		\shade[ball color=black!5] (   0,  1,  0) circle (1.2mm) node {$+q$};
		\shade[ball color=black!5] (   0, -1,  0) circle (1.2mm) node {$+q$};
		\shade[ball color=black!5] (   0,  0,  1) circle (1.2mm) node {$+q$};
		\shade[ball color=black!5] (   0,  0, -1) circle (1.2mm) node {$+q$};
		\shade[ball color=black!5] (  -1,  0,  0) circle (1.2mm) node {$-q$};
		\shade[ball color=black!5] (-0.5,  0,  0) circle (1.2mm) node {$-q$};
		\shade[ball color=black!5] (   1,  0,  0) circle (1.2mm) node {$-q$};
		\shade[ball color=black!5] (   2,  0,  0) circle (1.2mm) node {$-q$};

		\shade[ball color=black!50] (0, 0, 0) circle (.3mm) node[above] {$\vec 0$};
		\shade[ball color=black!50] (3/16, 0, 0) circle (.3mm) node[above] {$\vec s$};

		\draw[|-|] (-1, -1, +1) ++(-0.1, 0, 0) -- ++(0, 1, 0) node[midway, left] {$d$};

		\draw[->] (-1.5, 0, 0) -- ++(-0.5, 0, 0) node[midway, above] {Richtung $\vec p$};
	\end{tikzpicture}
	\caption{Anordnung der Ladungen. $\vec 0$ ist der Koordinatenursprung, $\vec s$ der Schwerpunkt der Ladungen.}
	\label{fig:ladungen}
\end{figure}

\subsection{Dipolmoment}

Das Dipolmoment $\vec p$ eines Dipols mit Ladung $+q$ und $-q$ und Separation
$\vec d$ ist:
\[
	\vec p = q \vec d
\]

Ich bestimme den Schwerpunkt:
\begin{align*}
	\vec s &= \frac 18 \sum_{i=1}^8 \vec r_i \\
		   &= \frac 18 \del{
		\begin{pmatrix}
			0 \\ d \\ 0
		\end{pmatrix}
		+
		\begin{pmatrix}
			0 \\ -d \\ 0
		\end{pmatrix}
		+
		\begin{pmatrix}
			0 \\ 0 \\ d
		\end{pmatrix}
		+
		\begin{pmatrix}
			0 \\ 0 \\ -d
		\end{pmatrix}
		+
		\begin{pmatrix}
			-d \\ 0 \\ 0
		\end{pmatrix}
		+
		\begin{pmatrix}
			-\half d \\ 0 \\ 0
		\end{pmatrix}
		+
		\begin{pmatrix}
			d \\ 0 \\ 0
		\end{pmatrix}
		+
		\begin{pmatrix}
			2d \\ 0 \\ 0
		\end{pmatrix}
	} \\
	&= \begin{pmatrix}
	\frac 3{16} d \\ 0 \\ 0
	\end{pmatrix}
\end{align*}

Bei mehreren Ladungen $\del{\del{q_i, \vec r_i}}_{i \in \set{1, \ldots, 8}}$ setzt sich das Dipolmoment aus den einzelnen Momenten zusammen. Dabei ist $\vec r_i$ die Position und $\vec s$ der Schwerpunkt der Ladungsverteilung:
\begin{align*}
	\vec p &= \sum_{i=1}^8 q_i \del{\vec r_i - \vec s} \\
		   &= q \del{
		\begin{pmatrix}
			-\frac 3{4}d \\ 0 \\ 0
		\end{pmatrix}
		-
		\begin{pmatrix}
			d -\frac 3{16}d \\ 0 \\ 0
		\end{pmatrix}
		-
		\begin{pmatrix}
			\half d-\frac 3{16}d  \\ 0 \\ 0
		\end{pmatrix}
		-
		\begin{pmatrix}
			 d-\frac 3{16} \\ 0 \\ 0
		\end{pmatrix}
		-
		\begin{pmatrix}
			 2d-\frac 3{16}d \\ 0 \\ 0
		\end{pmatrix}
	} \\
	&=
	q \begin{pmatrix}
		- \frac 92 d \\ 0 \\ 0
	\end{pmatrix}
\end{align*}

\subsection{Quadrupoltensor}

Das Quadrupolmoment ist gegegeben durch:
\begin{align*}
	Q_{ij} &= \int \dif{}^3 \vec x' \del{3 x_i' x_j' - r'^2 \delta_{ij}} \rho\del{\vec x'} \\
	\intertext{
		In diesem Fall hier habe ich keine kontinuierliche Ladungsverteilung,
		sondern diskrete Ladungen. Dies kann ich mit der $\delta$-Funktion
		modellieren:
	}
	&= \int \dif{}^3 \vec x' \del{3 x_i' x_j' - r'^2 \delta_{ij}} \sum_{k=1}^8 q_k \delta\del{\vec r_k - \vec x'} \\
	\intertext{
		Integral und Summe kann ich vertauschen. Das Integral über die
		$\delta$-Funktion liefert die gewünschten diskreten Ladungen.
	}
	&= \sum_{k=1}^8 \int \dif{}^3 \vec x' \del{3 x_i' x_j' - r'^2 \delta_{ij}} q_k \delta\del{\vec r_k - \vec x'} \\
	\intertext{
		Da die $\delta$-Distribution nur auf den 8 Stellen von 0 verschieden
		ist, kann ich das Integral ausführen. Mit $r_{ki} = \inner{\vec
		r_k}{\vec e_i}$:
	}
	Q_{ij} &= \sum_{k=1}^8 q_k \del{3 r_{ki} r_{kj} - \del{r_k}^2 \delta_{ij}} \\
	\intertext{
		Mit dem Tensorprodukt $\otimes$ kann ich das auch als ganzen Tensor
		schreiben. Dabei bezeichnet $\tens 1$ den Einheitstensor oder -matrix.
	}
	\tens Q &= \sum_{k=1}^8 q_k \del{3 \vec r_k \otimes \vec r_k - \tens 1 r_k^2}
	\intertext{
		Nun kann ich alle Zahlen einsetzen und erhalte:
	}
	\tens Q &= \frac {d^2q}{4}
	\begin{pmatrix}
		-246 & 27 & 27 \\
		  27 & 123 & 27 \\
		  27 & 27 & 123
	\end{pmatrix}
\end{align*}

Der Tensor ist symmetrisch und spurlos.

%%%%%%%%%%%%%%%%%%%%%%%%%%%%%%%%%%%%%%%%%%%%%%%%%%%%%%%%%%%%%%%%%%%%%%%%%%%%%%%
%                Magnetisches Feld eines ausgedehnten Leiters                 %
%%%%%%%%%%%%%%%%%%%%%%%%%%%%%%%%%%%%%%%%%%%%%%%%%%%%%%%%%%%%%%%%%%%%%%%%%%%%%%%

\section{Magnetisches Feld eines ausgedehnten Leiters}
\label{4}

Es gilt $\vnabla \times \vec H = \vec j$. Dies kann ich auch in Integralen
schreiben, für dieses Problem benutze ich Zylinderkoordinaten. Ich betrachte
eine Kreisfläche $A$ mit Radius $r$, die den Leiter in einem Kreis schneidet.
Das Normalenvektorfeld auf $A$ ist $\vec \nu$, der Tangentialvektor an den
Kreisrand ist $\vec \tau$. Die Stromdichte $\vec j$ ist $\frac{\vec I}{\pi R^2}$ Somit gilt nach dem Satz von Stokes:
%
\[
	\oint_{\partial A} \dif l \inner{\vec H}{\vec \tau} = \int_A \dif A \inner{\vec j}{\vec \nu}
\]

Die Integrale sind für den Fall $r > R$ durch Wahl des Koordinatensystems trivial. Somit folgt für das magnetische Feld:
\[
	2 \pi r H = I
	\quad \Leftrightarrow \quad
	H(r) = \frac{1}{2\pi} \frac{1}r I
\]

Falls $r < R$ gilt, ist der umschlossene Strom kleiner. Dann gilt nur noch:
\[
	2 \pi r H = \frac{r^2}{R^2} I
	\quad \Leftrightarrow \quad
	H(r) = \frac{1}{2\pi} \frac{r}{R^2} I
\]

Das Feld ist ein Wirbelfeld, dessen Richtung mit der rechten Handregel bestimmt
werden kann.

%%%%%%%%%%%%%%%%%%%%%%%%%%%%%%%%%%%%%%%%%%%%%%%%%%%%%%%%%%%%%%%%%%%%%%%%%%%%%%%
%                       geladene Walze, Rohr und Platte                       %
%%%%%%%%%%%%%%%%%%%%%%%%%%%%%%%%%%%%%%%%%%%%%%%%%%%%%%%%%%%%%%%%%%%%%%%%%%%%%%%

\section{geladene Walze, Rohr und Platte}
\label{5}

Bei all diesen Aufgaben benutze ich das Gesetz von Gauß.

\subsection{Hohlzylinder}

Die Oberflächenladungsdichte auf der Außenseite sei $\sigma$. Als
Integrationsvolumen benutze ich einen Zylinder mit Radius $r$, der konzentrisch mit dem
Hohlzylinder liegt. Dann gilt:
\[
	\oint_{\partial V} \dif A \inner{\vec D}{\vec \nu} = \int_V \dif V \rho
\]

Die Integrale sind hier recht trivial:
\[
	2 \pi r h D = 2 \pi R h \sigma
	\quad \Leftrightarrow \quad
	\vec D = \sigma \frac{R}{r} \hat{\vec r}
\]

Innerhalb des Zylinders befindet sich keine Ladung, so dass es dort auch kein Feld bilden kann.

Das Potential ist das negative Feld nach dem Radius integriert. Dabei ist es innerhalb des Zylinders konstant, weil keine Kraft wirkt.
\[
	\phi(r) = \begin{cases}
	-	\frac{1}{\epsilon_0} R \sigma \ln\del{r} & r \geq R \\
														0 & r < R
	\end{cases}
\]

Verschiebungsdichte und Potential sind in Abbildung \ref{fig:Dhohl} beziehungsweise \ref{fig:phihohl} gezeigt.

\begin{figure}[h]
	\centering
	\begin{minipage}[b]{0.45\textwidth}
		\centering
		\begin{tikzpicture}
			\draw[domain=1:3] plot (\x, {1 / \x});
			\draw[thin, ->] (0, -1) -- ++(3, 0) node[right] {$r$};
			\draw[thin, ->] (0, -1) -- ++(0, +1.5) node[above] {$D(r)$};
			\draw[thin] (1, -0.9) -- ++(0, -0.2) node[below] {$R$};
		\end{tikzpicture}
		\caption{Verschiebungsdichte des Hohlzylinders}
		\label{fig:Dhohl}
	\end{minipage}
	\begin{minipage}[b]{0.45\textwidth}
		\centering
		\begin{tikzpicture}
			\draw[domain=1:3] plot (\x, {ln(\x)});
			\draw (0, 0) -- (1, 0);
			\draw[thin, ->] (0, -1) -- ++(3, 0) node[right] {$r$};
			\draw[thin, ->] (0, -1) -- ++(0, +1.5) node[above] {$-\phi(r)$};
			\draw[thin] (1, -0.9) -- ++(0, -0.2) node[below] {$R$};
			\draw (0.1, 0) -- (-0.1, 0) node[left] {$0$};
		\end{tikzpicture}
		\caption{Potential des Hohlzylinders}
		\label{fig:phihohl}
	\end{minipage}
\end{figure}

\subsection{Vollzylinder}

Der Vollzylinder funktioniert analog für den Fall $r > R$. Nur wird die Fläche und Oberflächenladungsdichte durch Volumen und Ladungsdichte ersetzt.
\[
	2 \pi r h D = \pi R^2 h \rho
	\quad \Leftrightarrow \quad
	\vec D = \half \rho \frac{R^2}{r} \hat{\vec r}
\]

Innerhalb des Vollzylinders gibt es nun ein Feld. Dieses ist:
\[
	2 \pi r h D = \pi r^2 h \rho
	\quad \Leftrightarrow \quad
	\vec D = \half \rho r \hat{\vec r}
\]

Das Potential ist wieder das negative Feld nach dem Radius integriert. Es ergibt sich:
\[
	\phi(r) = \begin{cases}
		- \half \frac{1}{\epsilon_0} \rho R^2 \ln\del{r} & r \geq R \\
   - \frac 14 \frac{1}{\epsilon_0} \rho r^2 + \frac 14 \frac{1}{\epsilon_0} \rho R^2 + \phi(R) & r < R
	\end{cases}
\]

Verschiebungsdichte und Potential sind in Abbildung \ref{fig:Dvoll} beziehungsweise \ref{fig:phivoll} gezeigt.

\begin{figure}[h]
	\centering
	\begin{minipage}[b]{0.45\textwidth}
		\centering
		\begin{tikzpicture}
			\draw[domain=1:3] plot (\x, {0.5 / \x});
			\draw[domain=0:1] plot (\x, {0.5 * \x});
			\draw[thin, ->] (0, -1) -- ++(3, 0) node[right] {$r$};
			\draw[thin, ->] (0, -1) -- ++(0, +1.5) node[above] {$D(r)$};
			\draw[thin] (1, -0.9) -- ++(0, -0.2) node[below] {$R$};
		\end{tikzpicture}
		\caption{Verschiebungsdichte des Vollzylinders}
		\label{fig:Dvoll}
	\end{minipage}
	\begin{minipage}[b]{0.45\textwidth}
		\centering
		\begin{tikzpicture}
			\draw[domain=1:3] plot (\x, {0.5 * ln(\x)});
			\draw[domain=0:1] plot (\x, {0.25 * \x^2 - 0.25});
			\draw[thin, ->] (0, -1) -- ++(3, 0) node[right] {$r$};
			\draw[thin, ->] (0, -1) -- ++(0, +1.5) node[above] {$-\phi(r)$};
			\draw[thin] (1, -0.9) -- ++(0, -0.2) node[below] {$R$};
		\end{tikzpicture}
		\caption{Potential des Vollzylinders}
		\label{fig:phivoll}
	\end{minipage}
\end{figure}

\subsection{Platte}

Bei einer Platte wähle ich ein beliebiges Prisma, dessen Stirnflächen (Flächeninhalt $A$) parallel zur Platte sind und das von der Platte geschnitten wird. Aus der Symmetrie folgt dann, dass kein Fluss durch die Seitenflächen geht. Nach dem Gesetz von Gauß gilt dann:
\[
	2 A D = A \sigma
	\quad \Leftrightarrow \quad
	\vec D = \half \sigma \hat{\vec r}
\]

Das Potential ist das negative Feld nach dem Abstand integriert:
\[
	\phi(r) = - \half \frac{1}{\epsilon_0} \sigma \abs r
\]

Verschiebungsdichte und Potential sind in Abbildung \ref{fig:Dplatte} beziehungsweise \ref{fig:phiplatte} gezeigt.

\begin{figure}[h]
	\centering
	\begin{minipage}[b]{0.45\textwidth}
		\centering
		\begin{tikzpicture}
			\draw (-1, -1) -- (0, -1);
			\draw (1, 1) -- (0, 1);
			\draw[thin, ->] (-1, 0) -- (1, 0) node[right] {$r$};
			\draw[thin, ->] (0, -1.3) -- (0, +1.3) node[above] {$D(r)$};
		\end{tikzpicture}
		\caption{Verschiebungsdichte der Platte}
		\label{fig:Dplatte}
	\end{minipage}
	\begin{minipage}[b]{0.45\textwidth}
		\centering
		\begin{tikzpicture}
			\draw[domain=-1:1] plot (\x, {-abs(\x)});
			\draw[thin, ->] (-1, 0) -- (1, 0) node[right] {$r$};
			\draw[thin, ->] (0, 0) -- ++(0, +1) node[above] {$\phi(r)$};
		\end{tikzpicture}
		\caption{Potential der Platte}
		\label{fig:phiplatte}
	\end{minipage}
\end{figure}

\bibliography{../../zentrale_BibTeX/Central}
\bibliographystyle{plain}

\end{document}

% vim: spell spelllang=de
