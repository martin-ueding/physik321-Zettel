% Copyright © 2012 Martin Ueding <dev@martin-ueding.de>
%
\documentclass[11pt, ngerman, fleqn]{article}

\usepackage[a4paper, left=3cm, right=2cm, top=2cm, bottom=2cm]{geometry}
\usepackage[activate]{pdfcprot}
\usepackage[cdot, squaren]{SIunits}
\usepackage[iso]{isodate}
\usepackage[parfill]{parskip}
\usepackage[scaled]{beramono}
\usepackage[T1]{fontenc}
\usepackage[utf8]{inputenc}
\usepackage{amsmath}
\usepackage{amsthm}
\usepackage{babel}
\usepackage{color}
\usepackage{commath}
\usepackage{fancyhdr}
\usepackage{graphicx}
\usepackage{hyperref}
\usepackage{lastpage}
\usepackage{setspace}
\usepackage{tikz}

\usepackage[charter, greekuppercase=italicized]{mathdesign}

\definecolor{darkblue}{rgb}{0,0,.5}
\definecolor{darkgreen}{rgb}{0,.5,0}

\hypersetup{
	breaklinks=false,
	citecolor=darkgreen,
	colorlinks=true,
	linkcolor=black,
	menucolor=black,
	urlcolor=darkblue,
}

\setlength{\columnsep}{2cm}

\DeclareMathOperator{\arcsinh}{arsinh}
\DeclareMathOperator{\arsinh}{arsinh}
\DeclareMathOperator{\asinh}{arsinh}
\DeclareMathOperator{\card}{card}
\DeclareMathOperator{\diam}{diam}
\DeclareMathOperator{\re}{Re}

\newcommand{\curl}{\vnabla \times}
\newcommand{\dalambert}{\mathop{{}\Box}\nolimits}
\newcommand{\divergence}[1]{\inner{\vnabla}{#1}}
\newcommand{\ee}{\mathrm e}
\newcommand{\emesswert}{\del{\messwert \pm \messwert}}
\newcommand{\ev}{\hat{\vec e}}
\newcommand{\e}[1]{\cdot 10^{#1}}
\newcommand{\fehlt}{\textcolor{red}{Hier fehlen noch Inhalte.}}
\newcommand{\half}{\frac 12}
\newcommand{\ii}{\mathrm i}
\newcommand{\inner}[2]{\left\langle #1, #2 \right\rangle}
\newcommand{\laplace}{\mathop{{}\bigtriangleup}\nolimits}
\newcommand{\messwert}{\textcolor{blue}{\square}}
\newcommand{\punkte}{\textcolor{white}{xxxxx}}
\newcommand{\tens}[1]{\boldsymbol{\mathsf{#1}}}
\newcommand{\vnabla}{\vec \nabla}
\renewcommand{\vec}[1]{\boldsymbol{#1}}

\newcommand{\themodul}{physik321}
\newcommand{\thegruppe}{Gruppe 8 -- Julia Volmer}
\newcommand{\theuebung}{9}

\pagestyle{fancy}

\fancyfoot[C]{\footnotesize{\thegruppe}}
\fancyfoot[L]{\footnotesize{Martin Ueding, Simon Schlepphorst}}
\fancyfoot[R]{\footnotesize{Seite \thepage\ / \pageref{LastPage}}}
\fancyhead[L]{\themodul{} -- Übung \theuebung}

\def\thesection{H \theuebung.\arabic{section}}
\def\thesubsubsection{\thesubsection\alph{section}}

\title{\themodul{} -- Übung \theuebung \\ \vspace{0.5cm} \large{\thegruppe}}

\author{
	Martin Ueding \\ \small{\href{mailto:mu@uni-bonn.de}{mu@uni-bonn.de}}
	\and
	Simon Schlepphorst \\ \small{\href{mailto:s2@uni-bonn.de}{s2@uni-bonn.de}}
}

\begin{document}

\maketitle

\begin{table}[h]
	\centering
	\begin{tabular}{l|c|c|c|c}
		Aufgabe & \ref 1 & \ref 2 & \ref 3 & $\sum$   \\
		\hline
		Punkte & \punkte / 10 & \punkte / 10 & \punkte / 10 & \punkte / 30
	\end{tabular}
\end{table}

%%%%%%%%%%%%%%%%%%%%%%%%%%%%%%%%%%%%%%%%%%%%%%%%%%%%%%%%%%%%%%%%%%%%%%%%%%%%%%%
%                         inhomogene Wellengleichung                          %
%%%%%%%%%%%%%%%%%%%%%%%%%%%%%%%%%%%%%%%%%%%%%%%%%%%%%%%%%%%%%%%%%%%%%%%%%%%%%%%

\section{inhomogene Wellengleichung}
\label 1

Die interessanten Maxwellgleichungen im Vakuum (ohne Ladungen und Ströme) sind:
\[
	\curl \vec E + \dot{\vec B} = \vec 0
	,\quad
	\curl \vec B - \frac{1}{c^2} \dot{\vec E} = \vec 0
\]

Für die erste Wellengleichung, $\dalambert \vec E = \lambda$, bilde ich die Rotation der ersten Gleichung und leite die zweite nach der Zeit ab:
\[
	\curl \curl \vec E + \curl \dot{\vec B} = \vec 0
	,\quad
	\curl \dot{\vec B} - \frac{1}{c^2} \ddot{\vec E} = \vec 0
\]

Außerdem ist $\curl \curl \vec x = \vnabla \divergence{\vec x} - \laplace \vec x$. Wobei hier die Divergenzen alle $0$ sind.

Mit diesen Vorüberlegungen kann ich die erste Wellengleichung umformen:
\begin{align*}
	\dalambert \vec E &= \lambda_1 \\
	\laplace \vec E - \frac{1}{c^2} \ddot{\vec E} &= \lambda_1 \\
	- \curl \curl \vec E - \frac{1}{c^2} \ddot{\vec E} &= \lambda \\
	\intertext{%
		Wir setzen eine der Gleichungen ein.
	}
	- \curl \curl \vec E - \curl \dot{\vec B} &= \lambda_1 \\
	\intertext{%
		Die Rotation können wir aufheben. Dabei müssen wir einen beliebigen
		Gradienten addieren.
	}
	- \curl \vec E - \dot{\vec B} &= \lambda_1 + \vnabla \phi \\
	\intertext{%
		Die linke Seite ist gerade die negative Stromdichte, die hier
		allerdings $0$ ist.
	}
	0 &= \lambda_1 + \vnabla \phi \\
	\lambda_1 &= - \vnabla \phi
\end{align*}

Die Funktion $\lambda_1$ ist also ein beliebiger Gradient.

Analog bilden wir anders herum Rotation und Zeitableitung der beiden Gleichungen:
\[
	\curl \dot{\vec E} + \ddot{\vec B} = 0
	,\quad
	\curl \curl B - \curl \dot{\vec D} = 0
\]

Nun können wir wieder die Wellengleichung umformen:
\begin{align*}
	\dalambert \vec B &= \lambda_2 \\
	\laplace \vec B - \frac{1}{c^2} \ddot{\vec B} &= \lambda_2 \\
- \curl \curl \vec B - \frac{1}{c^2} \ddot{\vec B} &= \lambda_2 \\
	   - \curl \curl \vec B - \frac{1}{c^2} \curl \dot{\vec E} &= \lambda_2 \\
	\intertext{%
		Wir nehmen wieder eine Rotation weg und addieren einen Gradienten.
	}
	   - \curl \vec B - \frac{1}{c^2} \dot{\vec E} &= \lambda_2 + \vnabla \psi \\
	   0 &= \lambda_2 + \vnabla \psi \\
	   \lambda_2 &= - \vnabla \psi
\end{align*}

Die zweite Funktion $\lambda_2$ ist also wieder ein Gradient.

%%%%%%%%%%%%%%%%%%%%%%%%%%%%%%%%%%%%%%%%%%%%%%%%%%%%%%%%%%%%%%%%%%%%%%%%%%%%%%%
%                         zirkular polarisierte Welle                         %
%%%%%%%%%%%%%%%%%%%%%%%%%%%%%%%%%%%%%%%%%%%%%%%%%%%%%%%%%%%%%%%%%%%%%%%%%%%%%%%

\section{zirkular polarisierte Welle}
\label 2

%%%%%%%%%%%%%%%%%%%%%%%%%%%%%%%%%%%%%%%%%%%%%%%%%%%%%%%%%%%%%%%%%%%%%%%%%%%%%%%
%                         Coulomb- und Lorentzeichung                         %
%%%%%%%%%%%%%%%%%%%%%%%%%%%%%%%%%%%%%%%%%%%%%%%%%%%%%%%%%%%%%%%%%%%%%%%%%%%%%%%

\section{Coulomb- und Lorentzeichung}
\label 3

%\bibliography{../../zentrale_BibTeX/Central}
%\bibliographystyle{plain}

\end{document}

% vim: spell spelllang=de
