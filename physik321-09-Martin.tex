% Copyright © 2012 Martin Ueding <dev@martin-ueding.de>
%
\documentclass[11pt, ngerman, fleqn]{article}

\usepackage[a4paper, left=3cm, right=2cm, top=2cm, bottom=2cm]{geometry}
\usepackage[activate]{pdfcprot}
\usepackage[cdot, squaren]{SIunits}
\usepackage[iso]{isodate}
\usepackage[parfill]{parskip}
\usepackage[scaled]{beramono}
\usepackage[T1]{fontenc}
\usepackage[utf8]{inputenc}
\usepackage{amsmath}
\usepackage{amsthm}
\usepackage{babel}
\usepackage{color}
\usepackage{commath}
\usepackage{fancyhdr}
\usepackage{graphicx}
\usepackage{hyperref}
\usepackage{lastpage}
\usepackage{setspace}
\usepackage{tikz}

\usepackage[charter, greekuppercase=italicized]{mathdesign}

\definecolor{darkblue}{rgb}{0,0,.5}
\definecolor{darkgreen}{rgb}{0,.5,0}

\hypersetup{
	breaklinks=false,
	citecolor=darkgreen,
	colorlinks=true,
	linkcolor=black,
	menucolor=black,
	urlcolor=darkblue,
}

\setlength{\columnsep}{2cm}

\DeclareMathOperator{\arcsinh}{arsinh}
\DeclareMathOperator{\arsinh}{arsinh}
\DeclareMathOperator{\asinh}{arsinh}
\DeclareMathOperator{\card}{card}
\DeclareMathOperator{\diam}{diam}
\DeclareMathOperator{\re}{Re}

\newcommand{\curl}{\vnabla \times}
\newcommand{\dalambert}{\mathop{{}\Box}\nolimits}
\newcommand{\divergence}[1]{\inner{\vnabla}{#1}}
\newcommand{\ee}{\mathrm e}
\newcommand{\emesswert}{\del{\messwert \pm \messwert}}
\newcommand{\ev}{\hat{\vec e}}
\newcommand{\e}[1]{\cdot 10^{#1}}
\newcommand{\fehlt}{\textcolor{red}{Hier fehlen noch Inhalte.}}
\newcommand{\half}{\frac 12}
\newcommand{\ii}{\mathrm i}
\newcommand{\inner}[2]{\left\langle #1, #2 \right\rangle}
\newcommand{\laplace}{\mathop{{}\bigtriangleup}\nolimits}
\newcommand{\messwert}{\textcolor{blue}{\square}}
\newcommand{\punkte}{\textcolor{white}{xxxxx}}
\newcommand{\tens}[1]{\boldsymbol{\mathsf{#1}}}
\newcommand{\vnabla}{\vec \nabla}
\renewcommand{\vec}[1]{\boldsymbol{#1}}

\newcommand{\themodul}{physik321}
\newcommand{\thegruppe}{Gruppe 8 -- Julia Volmer}
\newcommand{\theuebung}{9}

\pagestyle{fancy}

\fancyfoot[C]{\footnotesize{\thegruppe}}
\fancyfoot[L]{\footnotesize{Martin Ueding, Simon Schlepphorst}}
\fancyfoot[R]{\footnotesize{Seite \thepage\ / \pageref{LastPage}}}
\fancyhead[L]{\themodul{} -- Übung \theuebung}

\def\thesection{H \theuebung.\arabic{section}}
\def\thesubsubsection{\thesubsection\alph{section}}

\title{\themodul{} -- Übung \theuebung \\ \vspace{0.5cm} \large{\thegruppe}}

\author{
	Martin Ueding \\ \small{\href{mailto:mu@uni-bonn.de}{mu@uni-bonn.de}}
	\and
	Simon Schlepphorst \\ \small{\href{mailto:s2@uni-bonn.de}{s2@uni-bonn.de}}
}

\begin{document}

\maketitle

\begin{table}[h]
	\centering
	\begin{tabular}{l|c|c|c|c}
		Aufgabe & \ref 1 & \ref 2 & \ref 3 & $\sum$   \\
		\hline
		Punkte & \punkte / 10 & \punkte / 10 & \punkte / 10 & \punkte / 30
	\end{tabular}
\end{table}

%%%%%%%%%%%%%%%%%%%%%%%%%%%%%%%%%%%%%%%%%%%%%%%%%%%%%%%%%%%%%%%%%%%%%%%%%%%%%%%
%                         inhomogene Wellengleichung                          %
%%%%%%%%%%%%%%%%%%%%%%%%%%%%%%%%%%%%%%%%%%%%%%%%%%%%%%%%%%%%%%%%%%%%%%%%%%%%%%%

\section{inhomogene Wellengleichung}
\label 1

Die interessanten Maxwellgleichungen im Vakuum (ohne Ladungen und Ströme) sind:
\[
	\curl \vec E + \dot{\vec B} = \vec 0
	,\quad
	\curl \vec B - \frac{1}{c^2} \dot{\vec E} = \vec 0
\]

Für die erste Wellengleichung, $\dalambert \vec E = \lambda$, bilde ich die Rotation der ersten Gleichung und leite die zweite nach der Zeit ab:
\[
	\curl \curl \vec E + \curl \dot{\vec B} = \vec 0
	,\quad
	\curl \dot{\vec B} - \frac{1}{c^2} \ddot{\vec E} = \vec 0
\]

Außerdem ist $\curl \curl \vec x = \vnabla \divergence{\vec x} - \laplace \vec x$. Wobei hier die Divergenzen alle $0$ sind.

Mit diesen Vorüberlegungen kann ich die erste Wellengleichung umformen:
\begin{align*}
	\dalambert \vec E &= \lambda_1 \\
	\laplace \vec E - \frac{1}{c^2} \ddot{\vec E} &= \lambda_1 \\
	- \curl \curl \vec E - \frac{1}{c^2} \ddot{\vec E} &= \lambda \\
	\intertext{%
		Wir setzen eine der Gleichungen ein.
	}
	- \curl \curl \vec E - \curl \dot{\vec B} &= \lambda_1 \\
	\intertext{%
		Die Rotation können wir aufheben. Dabei müssen wir einen beliebigen
		Gradienten addieren.
	}
	- \curl \vec E - \dot{\vec B} &= \lambda_1 + \vnabla \phi \\
	\intertext{%
		Die linke Seite ist gerade die negative Stromdichte, die hier
		allerdings $0$ ist.
	}
	0 &= \lambda_1 + \vnabla \phi \\
	\lambda_1 &= - \vnabla \phi
\end{align*}

Die Funktion $\lambda_1$ ist also ein beliebiger Gradient.

Analog bilden wir anders herum Rotation und Zeitableitung der beiden Gleichungen:
\[
	\curl \dot{\vec E} + \ddot{\vec B} = 0
	,\quad
	\curl \curl B - \curl \dot{\vec D} = 0
\]

Nun können wir wieder die Wellengleichung umformen:
\begin{align*}
	\dalambert \vec B &= \lambda_2 \\
	\laplace \vec B - \frac{1}{c^2} \ddot{\vec B} &= \lambda_2 \\
- \curl \curl \vec B - \frac{1}{c^2} \ddot{\vec B} &= \lambda_2 \\
	   - \curl \curl \vec B - \frac{1}{c^2} \curl \dot{\vec E} &= \lambda_2 \\
	\intertext{%
		Wir nehmen wieder eine Rotation weg und addieren einen Gradienten.
	}
	   - \curl \vec B - \frac{1}{c^2} \dot{\vec E} &= \lambda_2 + \vnabla \psi \\
	   0 &= \lambda_2 + \vnabla \psi \\
	   \lambda_2 &= - \vnabla \psi
\end{align*}

Die zweite Funktion $\lambda_2$ ist also wieder ein Gradient.

%%%%%%%%%%%%%%%%%%%%%%%%%%%%%%%%%%%%%%%%%%%%%%%%%%%%%%%%%%%%%%%%%%%%%%%%%%%%%%%
%                         zirkular polarisierte Welle                         %
%%%%%%%%%%%%%%%%%%%%%%%%%%%%%%%%%%%%%%%%%%%%%%%%%%%%%%%%%%%%%%%%%%%%%%%%%%%%%%%

\section{zirkular polarisierte Welle}
\label 2

\subsection{Magnetfeld}

Gegeben ist das elektrische Feld:
\[
	\vec E = \re\del{f(x-ct) \del{\ev_y + \ii \ev_z}}
\]

Wir bilden die Rotation dieses Feldes:
\[
	\curl \vec E = \re\del{f'(x-ct) \del{-\ii \ev_y + \ev_z}}
\]

Dies integrieren wir nach der Zeit und wählen alle Integrationskonstanten gleich $0$:
\[
	\vec B = \frac 1c \re\del{f(x-ct) \del{\ii \ev_y - \ev_z}}
\]

\subsection{Energiedichten}

Wir bestimmen die Energiedichte:
\begin{align*}
	\omega
	&= \half\del{\inner{\vec B}{\vec H} + \inner{\vec D}{\vec E}} \\
	&= \half f(x-ct)f^*(x-ct) \del{\frac{1}{c^2} \frac{1}{\mu_0} \del{\ii\ev_y - \ev_z}\del{-\ii\ev_y - \ev_z} + \varepsilon_0 \del{\ev_y+\ii\ev_z}\del{\ev_y - \ii \ev_z}} \\
	&= 2 \varepsilon_0 f(x-ct)f^*(x-ct) \\
\end{align*}

Und die Energiestromdichte:
\begin{align*}
	\vec S
	&= \vec E \times \vec H \\
	&= \frac{1}{c \mu} f(x-ct)f^*(x-ct) \begin{pmatrix} 0 \\ 1 \\ \ii \end{pmatrix} \times \begin{pmatrix} 0 \\ \ii \\ 1 \end{pmatrix} \\
	&= 2 \sqrt{\frac{\varepsilon_0}{\mu_0}} f(x-ct)f^*(x-ct) \ev_x
\end{align*}

Sowie den Energie-Impuls-Tensor $\tens T$. Dieser ist definiert als:
\[
	\tens T = \begin{pmatrix}
		\omega & S_x & S_y & S_z \\
	 S_x & G_{xx} & G_{xy} & G_{xz} \\
  S_y & G_{yx} & G_{yy} & G_{yz} \\
  S_z & G_{zx} & G_{zy} & G_{zz} \\
	\end{pmatrix} \\
\]

$\omega$ und $\vec S$ haben wir bereits bestimmt. Es fehlt der Spannungstensor
$\tens G$:
\[
	G^{\alpha\beta} = \omega \delta^{\alpha\beta} - \varepsilon_0 E^\alpha B^\beta - \frac{1}{\mu_0} B^\alpha B^\beta
\]

Oder als ganzer Tensor geschrieben:
\begin{align*}
	\tens G &= \omega \tens 1 - \varepsilon_0 \vec E \otimes \vec E - \frac{1}{\mu_0} \vec B \otimes \vec B \\
	&=
	\begin{pmatrix}
		\omega & 0 & 0 \\
				0 & \omega & 0 \\
		  0 & 0 & \omega
	\end{pmatrix}
	-
	\re\del{
		f^2(x-ct)
		\del{
			\varepsilon_0
			\begin{pmatrix}
				0 & 0 & 0 \\
		  0 & 1 & \ii \\
		  0 & \ii & -1
			\end{pmatrix}
			+
			\frac{1}{c^2 \mu_0}
			\begin{pmatrix}
				0 & 0 & 0 \\
		  0 & -1 & -\ii \\
		  0 & -\ii & 1
			\end{pmatrix}
		}
	} \\
	&=
	f(x-ct)f^*(x-ct) \varepsilon_0 \del{
		\begin{pmatrix}
			2 & 0 & 0 \\
			0 & 2 & 0 \\
			0 & 0 & 2
		\end{pmatrix}
		-
		\del{
			\begin{pmatrix}
				0 & 0 & 0 \\
		  0 & 1 & \ii \\
		  0 & \ii & -1
			\end{pmatrix}
			+
			\begin{pmatrix}
				0 & 0 & 0 \\
		  0 & -1 & -\ii \\
		  0 & -\ii & 1
			\end{pmatrix}
		}
	} \\
	&= 2 f(x-ct)f^*(x-ct) \varepsilon_0 \tens 1
\end{align*}

Die Diagonalform heißt wohl, dass dieses Feld nur Druck und keine Scherspannung
ausübt.

%%%%%%%%%%%%%%%%%%%%%%%%%%%%%%%%%%%%%%%%%%%%%%%%%%%%%%%%%%%%%%%%%%%%%%%%%%%%%%%
%                         Coulomb- und Lorenzeichung                         %
%%%%%%%%%%%%%%%%%%%%%%%%%%%%%%%%%%%%%%%%%%%%%%%%%%%%%%%%%%%%%%%%%%%%%%%%%%%%%%%

\section{Coulomb- und Lorenzeichung}
\label 3

\subsection{Begründung und Bestimmungsgleichungen}

Da $\divergence{\vec B} = 0$ gilt, kann ist dieses ein reines Wirbelfeld. Ein
solches kann immer als Rotation eines anderen Feldes, $\vec A$ geschrieben
werden:
\[
	\vec B = \curl \vec A
\]

Für den elektrostatischen Fall gilt $\curl \vec E = 0$, so dass sich dieses als
Gradient eines skalaren Feldes, $\varphi + c_1$ schreiben lässt:
\[
	\vec E = - \vnabla \varphi
\]

Für den elektrodynamischen, aber ladungsfreien Fall gilt $\divergence{\vec E} =
0$. Außerdem gilt $\curl \vec E = - \dot{\vec B}$. Wir setzen die Induktion ein
und erhalten:
\[
	\curl \vec E = - \curl \dot{\vec A}
	\iff
	\vec E = - \dot{\vec A} - \vnabla \dot \psi
\]

Durch die Linearität der Differentialgleichung können wir beide Fälle zusammensetzen und erhalten:
\[
	\vec E = - \vnabla \phi - \dot{\vec A}
\]

\subsection{Eichinvarianz der Potentiale}

In der vorherigen Aufgabe haben wir bei der Integration schon die Freiheiten
$c_1$ und $\vnabla \psi$ eingefügt. Das Vektorpotential $\vec A$ ist nur bis
auf einen Gradienten bestimmt, das skalare Potential $\varphi$ nur bis auf eine
additive Konstante.

\subsection{Coulombeichung}

Wir betrachten die inhomogenen Maxwellgleichungen:
\[
	\divergence{\vec E} = \frac{\rho}{\varepsilon}
	,\quad
	\curl \vec B - \frac{1}{c^2} \dot{\vec E} = \mu \vec j
\]

Dort setzen wir unsere Potentiale ein und erhalten nach einigen Umformungen folgendes Gleichungssystem:
\[
	\vnabla \del{\divergence{\vec A} + \frac{1}{c^2} \dot \varphi} - \dalambert \vec A = - \mu \vec j
	,\quad
	\laplace \varphi + \divergence{\dot{\vec A}} = - \frac{\rho}{\varepsilon}
\]

Nun setzen wir $\divergence{\vec A} = 0$ ein und vereinfachen das System zu:
\[
	- \frac{1}{c^2} \vnabla \dot \varphi + \dalambert \vec A = - \mu \vec j
	,\quad
	\laplace \varphi = - \frac{\rho}{\varepsilon}
\]

Die zweite Gleichung ist gerade die aus der Elektrostatik. Diese können wir mit einem Poissonintegral lösen und erhalten:
\begin{equation}
	\label{eq:phi}
	\varphi = \frac{1}{4\pi\varepsilon_0} \int \dif{^3 r'} \frac{\rho}{\abs{\vec r - \vec r'}}
\end{equation}

Die erste Gleichung formen wir um und erhalten eine Gleichung für das $\vec A$-Feld:
\[
	\dalambert \vec A = - \mu \vec j + \frac{1}{c^2} \vnabla \dot \varphi 
\]

Wir setzen \eqref{eq:phi} in die obige Gleichung ein und erhalten:
\[
	\dalambert \vec A = - \mu \vec j - \frac{\mu_0}{4 \pi} \vnabla \int \dif{^3 r'} \frac{\dot \varphi}{\abs{\vec r - \vec r'}}
\]

Nun benutzen wir die Kontinuitätsgleichung. Diese lautet:
\[
	\dot \rho + \divergence{\vec j} = 0
	\iff
	\varepsilon \vnabla \dot \varphi = \divergence{\vec j}
\]

Unsere Gleichung wird zu:
\[
	\dalambert \vec A = - \mu \vec j - \frac{\mu_0}{4 \pi} \vnabla \int \dif{^3 r'} \frac{\divergence{\dot{\vec j}}}{\abs{\vec r - \vec r'}}
\]

Die Differenz auf der rechten Seite können wir als Rotation schreiben. Ein Vektorfeld lässt sich in ein Quellen- und ein Wirbelfeld zerlegen. Dies ist der erste Summand. Davon ziehen wir den Quellenteil ab, es bleibt der Wirbelteil.
\[
	\dalambert \vec A = - \frac{\mu_0}{4 \pi} \curl \int \dif{^3 r'} \frac{\divergence{\dot{\vec j}}}{\abs{\vec r - \vec r'}}
\]

\subsection{Lorenzeichung}

\fehlt

%\bibliography{../../zentrale_BibTeX/Central}
%\bibliographystyle{plain}

\end{document}

% vim: spell spelllang=de
