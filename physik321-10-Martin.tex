% Copyright © 2012 Martin Ueding <dev@martin-ueding.de>
%
% Copyright © 2012 Martin Ueding <dev@martin-ueding.de>
%
\documentclass[11pt, ngerman, fleqn]{scrartcl}

\usepackage{graphicx}

%%%%%%%%%%%%%%%%%%%%%%%%%%%%%%%%%%%%%%%%%%%%%%%%%%%%%%%%%%%%%%%%%%%%%%%%%%%%%%%
%                                Locale, date                                 %
%%%%%%%%%%%%%%%%%%%%%%%%%%%%%%%%%%%%%%%%%%%%%%%%%%%%%%%%%%%%%%%%%%%%%%%%%%%%%%%

\usepackage{babel}
\usepackage[iso]{isodate}

%%%%%%%%%%%%%%%%%%%%%%%%%%%%%%%%%%%%%%%%%%%%%%%%%%%%%%%%%%%%%%%%%%%%%%%%%%%%%%%
%                          Margins and other spacing                          %
%%%%%%%%%%%%%%%%%%%%%%%%%%%%%%%%%%%%%%%%%%%%%%%%%%%%%%%%%%%%%%%%%%%%%%%%%%%%%%%

\usepackage[activate]{pdfcprot}
\usepackage[left=3cm, right=2cm, top=2cm, bottom=2cm]{geometry}
\usepackage[parfill]{parskip}
\usepackage{setspace}

\setlength{\columnsep}{2cm}

%%%%%%%%%%%%%%%%%%%%%%%%%%%%%%%%%%%%%%%%%%%%%%%%%%%%%%%%%%%%%%%%%%%%%%%%%%%%%%%
%                                    Color                                    %
%%%%%%%%%%%%%%%%%%%%%%%%%%%%%%%%%%%%%%%%%%%%%%%%%%%%%%%%%%%%%%%%%%%%%%%%%%%%%%%

\usepackage{color}

\definecolor{darkblue}{rgb}{0,0,.5}
\definecolor{darkgreen}{rgb}{0,.5,0}
\definecolor{darkred}{rgb}{.7,0,0}

%%%%%%%%%%%%%%%%%%%%%%%%%%%%%%%%%%%%%%%%%%%%%%%%%%%%%%%%%%%%%%%%%%%%%%%%%%%%%%%
%                         Font and font like settings                         %
%%%%%%%%%%%%%%%%%%%%%%%%%%%%%%%%%%%%%%%%%%%%%%%%%%%%%%%%%%%%%%%%%%%%%%%%%%%%%%%

\usepackage[charter, greekuppercase=italicized]{mathdesign}
\usepackage{beramono}
\usepackage{berasans}

% Style of vectors and tensors.
\newcommand{\tens}[1]{\boldsymbol{\mathsf{#1}}}
\renewcommand{\vec}[1]{\boldsymbol{#1}}

%%%%%%%%%%%%%%%%%%%%%%%%%%%%%%%%%%%%%%%%%%%%%%%%%%%%%%%%%%%%%%%%%%%%%%%%%%%%%%%
%                               Input encoding                                %
%%%%%%%%%%%%%%%%%%%%%%%%%%%%%%%%%%%%%%%%%%%%%%%%%%%%%%%%%%%%%%%%%%%%%%%%%%%%%%%

\usepackage[T1]{fontenc}
\usepackage[utf8]{inputenc}

%%%%%%%%%%%%%%%%%%%%%%%%%%%%%%%%%%%%%%%%%%%%%%%%%%%%%%%%%%%%%%%%%%%%%%%%%%%%%%%
%                         Hyperrefs and PDF metadata                          %
%%%%%%%%%%%%%%%%%%%%%%%%%%%%%%%%%%%%%%%%%%%%%%%%%%%%%%%%%%%%%%%%%%%%%%%%%%%%%%%

\usepackage{hyperref}
\usepackage{lastpage}

\hypersetup{
	breaklinks=false,
	citecolor=darkgreen,
	colorlinks=true,
	linkcolor=black,
	menucolor=black,
	pdfauthor={Martin Ueding},
	urlcolor=darkblue,
}

%%%%%%%%%%%%%%%%%%%%%%%%%%%%%%%%%%%%%%%%%%%%%%%%%%%%%%%%%%%%%%%%%%%%%%%%%%%%%%%
%                               Math Operators                                %
%%%%%%%%%%%%%%%%%%%%%%%%%%%%%%%%%%%%%%%%%%%%%%%%%%%%%%%%%%%%%%%%%%%%%%%%%%%%%%%

\usepackage[thinspace, squaren]{SIunits}
\usepackage{amsmath}
\usepackage{amsthm}
\usepackage{commath}

% Word like operators.
\DeclareMathOperator{\acosh}{arcosh}
\DeclareMathOperator{\arcosh}{arcosh}
\DeclareMathOperator{\arcsinh}{arsinh}
\DeclareMathOperator{\arsinh}{arsinh}
\DeclareMathOperator{\asinh}{arsinh}
\DeclareMathOperator{\card}{card}
\DeclareMathOperator{\diam}{diam}
\renewcommand{\Im}{\mathop{{}\mathrm{Im}}\nolimits}
\renewcommand{\Re}{\mathop{{}\mathrm{Re}}\nolimits}

% Special single letters.
\DeclareMathOperator{\fourier}{\mathcal{F}}
\newcommand{\C}{\ensuremath{\mathbb C}}
\newcommand{\ee}{\mathrm e}
\newcommand{\ii}{\mathrm i}
\newcommand{\N}{\ensuremath{\mathbb N}}
\newcommand{\R}{\ensuremath{\mathbb R}}
\newcommand{\Z}{\ensuremath{\mathbb Z}}

% Shape like operators.
\DeclareMathOperator{\dalambert}{\Box}
\DeclareMathOperator{\laplace}{\bigtriangleup}
\newcommand{\curl}{\vnabla \times}
\newcommand{\divergence}[1]{\inner{\vnabla}{#1}}
\newcommand{\vnabla}{\vec \nabla}

% Shortcuts
\newcommand{\ev}{\hat{\vec e}}
\newcommand{\e}[1]{\cdot 10^{#1}}
\newcommand{\half}{\frac 12}
\newcommand{\inner}[2]{\left\langle #1, #2 \right\rangle}

% Placeholders.
\newcommand{\emesswert}{\del{\messwert \pm \messwert}}
\newcommand{\fehlt}{\textcolor{darkred}{Hier fehlen noch Inhalte.}\marginpar{\textcolor{darkred}{!}}}
\newcommand{\messwert}{\textcolor{blue}{\square}}
\newcommand{\punkte}{\textcolor{white}{xxxxx}}

%%%%%%%%%%%%%%%%%%%%%%%%%%%%%%%%%%%%%%%%%%%%%%%%%%%%%%%%%%%%%%%%%%%%%%%%%%%%%%%
%                                  Headings                                   %
%%%%%%%%%%%%%%%%%%%%%%%%%%%%%%%%%%%%%%%%%%%%%%%%%%%%%%%%%%%%%%%%%%%%%%%%%%%%%%%

\usepackage{scrpage2}

\pagestyle{scrheadings}

\automark{section}
\cfoot{\footnotesize{Seite \thepage\ / \pageref{LastPage}}}
\chead{}
\ihead{}
\ohead{\rightmark}
\setheadsepline{.4pt}


\usepackage{tikz}

\newcommand{\themodul}{physik321}
\newcommand{\thegruppe}{Gruppe 8 -- Julia Volmer}
\newcommand{\theuebung}{10}

\ifoot{\footnotesize{Martin Ueding, Simon Schlepphorst}}
\ihead{\themodul{} -- Übung \theuebung}
\ofoot{\footnotesize{\thegruppe}}

\def\thesection{H \theuebung.\arabic{section}}
\def\thesubsubsection{\thesubsection\alph{section}}

\title{\themodul{} -- Übung \theuebung \\ \vspace{0.5cm} \large{\thegruppe}}

\author{
	Martin Ueding \\ \small{\href{mailto:mu@uni-bonn.de}{mu@uni-bonn.de}}
	\and
	Simon Schlepphorst \\ \small{\href{mailto:s2@uni-bonn.de}{s2@uni-bonn.de}}
}

\hypersetup{
	pdftitle={\themodul {} - Übung \theuebung},
}

\begin{document}

\maketitle

\begin{table}[h]
	\centering
	\begin{tabular}{l|c|c|c}
		Aufgabe & \ref 1 & \ref 2 & $\sum$   \\
		\hline
		Punkte & \punkte / 20 & \punkte / 20 & \punkte / 40
	\end{tabular}
\end{table}

%%%%%%%%%%%%%%%%%%%%%%%%%%%%%%%%%%%%%%%%%%%%%%%%%%%%%%%%%%%%%%%%%%%%%%%%%%%%%%%
%            elektrische Quadrupol- und magnetische Dipolstrahlung            %
%%%%%%%%%%%%%%%%%%%%%%%%%%%%%%%%%%%%%%%%%%%%%%%%%%%%%%%%%%%%%%%%%%%%%%%%%%%%%%%

\section{elektrische Quadrupol- und magnetische Dipolstrahlung}
\label 1

\fehlt

%%%%%%%%%%%%%%%%%%%%%%%%%%%%%%%%%%%%%%%%%%%%%%%%%%%%%%%%%%%%%%%%%%%%%%%%%%%%%%%
%                    Strahlung einer langsamen Punktladung                    %
%%%%%%%%%%%%%%%%%%%%%%%%%%%%%%%%%%%%%%%%%%%%%%%%%%%%%%%%%%%%%%%%%%%%%%%%%%%%%%%

\section{Strahlung einer langsamen Punktladung}
\label 2

\subsection{Ladungs- und Stromdichte}

Wir haben eine Punktladung mit Ladung $q$ an der Stelle $\vec r_0(t)$. Somit
ist die Ladungsdichte:
\[
	\rho\del{\vec r, t} = q \delta^3\del{\vec r - \vec r_0(t)}
\]

Es fließt nur an der Stelle $\vec r_0$ Strom, so dass die Stromdichte
sinnvollerweise ist:
\[
	\vec j\del{\vec r, t} = \rho\del{\vec r, t} \vec v(t)
\]

\subsection{Potentiale in Lorenz-Eichung}

\fehlt

\subsection{Näherungen}

Durch die Näherung $r \gg r_0$ werden die $\vec r - \vec r_0(t')$ Terme zu
$\vec r$. Wir können außerdem die Geschwindigkeit zur Zeit $t'$ direkt durch
$\beta_\text{ret}$ ausdrücken.
\begin{align*}
	\Phi\del{\vec r, t}
	&= \frac{q}{4 \pi \varepsilon_0} \frac{1}{r} \frac{1}{1 - \inner{\frac{\vec r}{r}}{\vec \beta_\text{ret}}} \\
	\intertext{%
		Oder mit Summenkonvention:
	}
	&= \frac{q}{4 \pi \varepsilon_0} \frac{1}{r} \frac{1}{1 - \frac{r_b}{r}\beta_\text{ret}^b} \\
	\intertext{%
		Diesen Ausdruck können wir nun wegen $v \ll c$ um $\vec
		\beta_\text{ret} = \vec 0$ entwickeln. Den Term nullter Ordnung lesen
		wir ab, es ist $1/r$. Für den Term erster Ordnung bilden wir die erste
		Ableitung nach $\beta^d$ (ohne Vorfaktor):
		\[
			\frac{1}{r} \frac{\frac{r_b}{r}}{\del{1 - \frac{r_d}{r}\beta_\text{ret}^b}^2}
		\]
		Wenn wir dies an der Stelle $\vec \beta_\text{ret} = \vec 0$ auswerten,
		erhalten wir nur $r_d/r^2$. Mit allen Komponenten $d$ zusammen erhalten
		wir den Vektor $\vec r$. Somit ist die Entwicklung:
	}
	&= \frac{q}{4 \pi \varepsilon_0} \del{\frac 1r + \frac{r_b}{r^2} \beta_\text{ret}^b} + \mathcal O\del{\vec \beta_\text{ret}^2} \\
	&= \frac{q}{4 \pi \varepsilon_0} \del{\frac 1r + \frac{1}{r^2}\inner{\vec r}{\vec \beta_\text{ret}}} + \mathcal O\del{\vec \beta_\text{ret}^2}
\end{align*}

\fehlt

\subsection{Felder}

Das elektrische Feld ist der negative Gradient des skalaren Potentials minus der Zeitableitung des $\vec A$-Feldes, so dass gilt:
\[
	\vec E = - \vnabla \Phi - \dot{\vec A}
\]

Wir bilden den zuerst den Gradienten des entwickelten skalaren Potentials:
\begin{align*}
	\vec E
	&= - \vnabla \Phi \\
	&= - \frac{q}{4 \pi \varepsilon_0} \vnabla \del{\frac 1r + \frac{1}{r^2}\inner{\vec r}{\vec \beta_\text{ret}} + \mathcal O\del{\vec \beta_\text{ret}^2}} \\ 
	\intertext{%
		Dabei werden wir den $\vnabla$-Operator in Kugelkoordinaten benutzen.
	}
	&= - \frac{q}{4 \pi \varepsilon_0} \del{\ev^r \dpd{}r + \frac 1r \ev^\theta \dpd{}\theta + \frac{1}{r \sin \theta} \ev^\phi \dpd{}\phi} \del{\frac 1r + \frac{1}{r^2}\inner{\vec r}{\vec \beta_\text{ret}} + \mathcal O\del{\vec \beta_\text{ret}^2}} \\ 
	&= - \frac{q}{4 \pi \varepsilon_0} \del{ - \frac 1{r^2} \ev^r + \del{\ev^r \dpd{}r + \frac 1r \ev^\theta \dpd{}\theta + \frac{1}{r \sin \theta} \ev^\phi \dpd{}\phi} \del{\frac{1}{r^2}\inner{\vec r}{\vec \beta_\text{ret}} + \mathcal O\del{\vec \beta_\text{ret}^2}}} \\ 
	\intertext{%
		Terme der Ordnung $1/r^2$ lassen wir fallen.
	}
	&= - \frac{q}{4 \pi \varepsilon_0} \del{\ev^r \dpd{}r + \frac 1r \ev^\theta \dpd{}\theta + \frac{1}{r \sin \theta} \ev^\phi \dpd{}\phi} \frac{1}{r^2}\inner{\vec r}{\vec \beta_\text{ret}} + \mathcal O\del{\frac 1{r^2}} \\ 
	&= - \frac{q}{4 \pi \varepsilon_0} \del{\ev^r \dpd{}r + \frac 1r \ev^\theta \dpd{}\theta + \frac{1}{r \sin \theta} \ev^\phi \dpd{}\phi} \frac{1}{r^2} \ev_a \beta_\text{ret}^a + \mathcal O\del{\frac 1{r^2}} \\ 
	%
	\intertext{%
		\fehlt
	}
	\intertext{%
		Den Summanden $\dot{\vec \beta}_\text{ret}$ erhalten wir durch
		$\dot{\vec A}$.
	}
	&= \frac{q}{4 \pi \varepsilon_0} \frac 1r \del{\dot{\beta}_\text{ret}^r \ev^r + \dot{\vec \beta}_\text{ret}} + \mathcal O\del{\frac 1{r^2}} \\
	&= \frac{q}{4 \pi \varepsilon_0} \frac 1r \del{\ev^r \inner{\ev^r}{\dot{\vec \beta}_\text{ret}} + \dot{\vec \beta}_\text{ret}} + \mathcal O\del{\frac 1{r^2}} \\
&= \frac{q}{4 \pi \varepsilon_0} \frac 1r \del{\ev^r \inner{\ev^r}{\dot{\vec \beta}_\text{ret}} + \dot{\vec \beta}_\text{ret} \inner{\ev^r}{\ev^r}} + \mathcal O\del{\frac 1{r^2}} \\
	&= \frac{q}{4 \pi \varepsilon_0} \frac 1r \ev^r \times \del{\ev^r \times \dot{\vec \beta}_\text{ret}} + \mathcal O\del{\frac 1{r^2}}
\end{align*}

\fehlt

\subsection{Polarisierung}

\fehlt

%\bibliography{../../zentrale_BibTeX/Central}
%\bibliographystyle{plain}

\end{document}

% vim: spell spelllang=de
